\documentclass[journal,12pt,twocolumn]{IEEEtran}

\usepackage{setspace}
\usepackage{gensymb}
\singlespacing
\usepackage[cmex10]{amsmath}

\usepackage{amsthm}

\usepackage{mathrsfs}
\usepackage{txfonts}
\usepackage{stfloats}
\usepackage{bm}
\usepackage{cite}
\usepackage{cases}
\usepackage{subfig}

\usepackage{longtable}
\usepackage{multirow}

\usepackage{enumitem}
\usepackage{mathtools}
\usepackage{steinmetz}
\usepackage{tikz}
\usepackage{circuitikz}
\usepackage{verbatim}
\usepackage{tfrupee}
\usepackage[breaklinks=true]{hyperref}
\usepackage{graphicx}
\usepackage{tkz-euclide}

\usetikzlibrary{calc,math}
\usepackage{listings}
    \usepackage{color}                                            %%
    \usepackage{array}                                            %%
    \usepackage{longtable}                                        %%
    \usepackage{calc}                                             %%
    \usepackage{multirow}                                         %%
    \usepackage{hhline}                                           %%
    \usepackage{ifthen}                                           %%
    \usepackage{lscape}     
\usepackage{multicol}
\usepackage{chngcntr}

\DeclareMathOperator*{\Res}{Res}

\renewcommand\thesection{\arabic{section}}
\renewcommand\thesubsection{\thesection.\arabic{subsection}}
\renewcommand\thesubsubsection{\thesubsection.\arabic{subsubsection}}

\renewcommand\thesectiondis{\arabic{section}}
\renewcommand\thesubsectiondis{\thesectiondis.\arabic{subsection}}
\renewcommand\thesubsubsectiondis{\thesubsectiondis.\arabic{subsubsection}}


\hyphenation{op-tical net-works semi-conduc-tor}
\def\inputGnumericTable{}                                 %%

\lstset{
%language=C,
frame=single, 
breaklines=true,
columns=fullflexible
}
\begin{document}


\newtheorem{theorem}{Theorem}[section]
\newtheorem{problem}{Problem}
\newtheorem{proposition}{Proposition}[section]
\newtheorem{lemma}{Lemma}[section]
\newtheorem{corollary}[theorem]{Corollary}
\newtheorem{example}{Example}[section]
\newtheorem{definition}[problem]{Definition}

\newcommand{\BEQA}{\begin{eqnarray}}
\newcommand{\EEQA}{\end{eqnarray}}
\newcommand{\R}{\mathbb{R}}
\newcommand{\define}{\stackrel{\triangle}{=}}
\bibliographystyle{IEEEtran}
\raggedbottom
\setlength{\parindent}{0pt}
\providecommand{\mbf}{\mathbf}
\providecommand{\pr}[1]{\ensuremath{\Pr\left(#1\right)}}
\providecommand{\qfunc}[1]{\ensuremath{Q\left(#1\right)}}
\providecommand{\sbrak}[1]{\ensuremath{{}\left[#1\right]}}
\providecommand{\lsbrak}[1]{\ensuremath{{}\left[#1\right.}}
\providecommand{\rsbrak}[1]{\ensuremath{{}\left.#1\right]}}
\providecommand{\brak}[1]{\ensuremath{\left(#1\right)}}
\providecommand{\lbrak}[1]{\ensuremath{\left(#1\right.}}
\providecommand{\rbrak}[1]{\ensuremath{\left.#1\right)}}
\providecommand{\cbrak}[1]{\ensuremath{\left\{#1\right\}}}
\providecommand{\lcbrak}[1]{\ensuremath{\left\{#1\right.}}
\providecommand{\rcbrak}[1]{\ensuremath{\left.#1\right\}}}
\theoremstyle{remark}
\newtheorem{rem}{Remark}
\newcommand{\sgn}{\mathop{\mathrm{sgn}}}
\providecommand{\abs}[1]{\ensuremath{\left\vert#1\right\vert}}
\providecommand{\res}[1]{\ensuremath{\Res\displaylimits_{#1}}} 
\providecommand{\norm}[1]{\ensuremath{\left\lVert#1\right\rVert}}
%\providecommand{\norm}[1]{\lVert#1\rVert}
\providecommand{\mtx}[1]{\mathbf{#1}}
\providecommand{\mean}[1]{E\ensuremath{\left[ #1 \right]}}
\providecommand{\fourier}{\overset{\mathcal{F}}{ \rightleftharpoons}}
%\providecommand{\hilbert}{\overset{\mathcal{H}}{ \rightleftharpoons}}
\providecommand{\system}{\overset{\mathcal{H}}{ \longleftrightarrow}}
	%\newcommand{\solution}[2]{\textbf{Solution:}{#1}}
\newcommand{\solution}{\noindent \textbf{Solution: }}
\newcommand{\cosec}{\,\text{cosec}\,}
\providecommand{\dec}[2]{\ensuremath{\overset{#1}{\underset{#2}{\gtrless}}}}
\newcommand{\myvec}[1]{\ensuremath{\begin{pmatrix}#1\end{pmatrix}}}
\newcommand{\mydet}[1]{\ensuremath{\begin{vmatrix}#1\end{vmatrix}}}
\numberwithin{equation}{subsection}

\makeatletter
\@addtoreset{figure}{problem}
\makeatother
\let\StandardTheFigure\thefigure
\let\vec\mathbf

\renewcommand{\thefigure}{\theproblem}

\def\putbox#1#2#3{\makebox[0in][l]{\makebox[#1][l]{}\raisebox{\baselineskip}[0in][0in]{\raisebox{#2}[0in][0in]{#3}}}}
     \def\rightbox#1{\makebox[0in][r]{#1}}
     \def\centbox#1{\makebox[0in]{#1}}
     \def\topbox#1{\raisebox{-\baselineskip}[0in][0in]{#1}}
     \def\midbox#1{\raisebox{-0.5\baselineskip}[0in][0in]{#1}}
\vspace{3cm}
\title{EE5609: Matrix Theory\\Assignment-9}
\author{Major Saurabh Joshi\\MTech Artificial Intelligence\\AI20MTECH13002 }
\maketitle

%\tableofcontents
\bigskip
\renewcommand{\thefigure}{\theenumi}
\renewcommand{\thetable}{\theenumi}
\begin{abstract}
This document explains how to find the standard basis vector. 
\end{abstract}
Download the latex code from 
%
%
%
\begin{lstlisting}
https://github.com/saurabh13002/EE5609/tree/master/Assignment9
\end{lstlisting}
%
\section{Problem}
Show that the vectors 
\begin{align}
& \vec{\alpha_1} = \myvec{1 & 0 & -1} & \vec{\alpha_2} = \myvec{1 & 2 & 1 } \\
& \vec{\alpha_3} = \myvec{0 & -3 & 2}
\end{align}
form a basis for $\R^3$. Express each of the standard basis vectors as linear combinations
of $\myvec{\vec{\alpha_1} &  \vec{\alpha_2} & \vec{\alpha_3}}$
\section{Theorem}
\begin{theorem} \label{1}
Let $\vec{V}$ be an $n$-dimensional vector space over the field $\vec{F}$, and let $\beta$ and $\beta'$ be two ordered basis of $\vec{V}$. Then, there is a unique, necessarily invertible, $n\times n$ matrix $\vec{P}$ with entries in $\vec{F}$ such that 
\begin{enumerate}
	\item $\begin{bmatrix}
	\vec{\alpha}
	\end{bmatrix}$$_\beta$ = $\vec{P}$  $\begin{bmatrix}
	\vec{\alpha} 
\end{bmatrix}$$_{\beta^{'}}$
    \item $\begin{bmatrix}
    	\vec{\alpha}
    \end{bmatrix}$$_{\beta^{'}}$= $\vec{P^{-1}}$$\begin{bmatrix}
    \vec{\alpha}	
\end{bmatrix}$$_\beta$
\end{enumerate}
for every vector $\vec{\alpha}$ in $\vec{V}$. The columns of $\vec{P}$ are given by
\begin{align}
\vec{P_j} =  \begin{bmatrix}
	\vec{\alpha_j}
\end{bmatrix}_{\beta}	\qquad j = 1,2,...,n
\end{align}
\end{theorem}
\section{Solution}
In order to show that the set of vectors $\alpha_1$, $\alpha_2$,and $\alpha_3$  are basis for $\R^3$. We first show that $\alpha_1$, $\alpha_2$,and $\alpha_3$ a  are linearly independent in $\R^3$ and also they span $\R^3$. Consider,
\begin{align}
& \vec{A} = \myvec{\vec{\alpha_1}^T & \vec{\alpha_2}^T & \vec{\alpha_3}^T } \\
& \vec{A} = \myvec{1 & 1 &  0 \\ 0 & 2 & -3  \\ -1 & 1 & 2 }\label{eqA}
\end{align}
Now,by row reduction
\begin{align}
\myvec{1 & 1 &  0 \\ 0 & 2 & -3  \\ -1 & 1 & 2} &\xleftrightarrow[]{R_3=R_3+R_1}\myvec{1 & 1 &  0 \\ 0 & 2 & -3  \\ 0 & 2 & 2  } \\
 &\xleftrightarrow[]{R_3=R_3-R_2}\myvec{1 & 1 &  0 \\ 0 & 2 & -3  \\ 0 & 0 & 5  } \\
&\xleftrightarrow[]{R_2=\frac{R_2}{2}}\myvec{1 & 1 &  0 \\ 0 & 1 &\frac{-3}{2}   \\ 0 & 0 & 5  } \\
&\xleftrightarrow[]{R_1=R_1-R_2}\myvec{1 & 0 &  \frac{3}{2} \\ 0 & 1 &\frac{-3}{2}   \\ 0 & 0 & 5  } \\
&\xleftrightarrow[]{R_3=\frac{R_3}{5}}\myvec{1 & 0 &   \frac{3}{2} \\ 0 & 1 & \frac{-3}{2}   \\ 0 & 0 & 1  } \\
&\xleftrightarrow[]{R_1=R_1-\frac{3}{2}R_3}\myvec{1 & 0 &  0 \\ 0 & 1 & \frac{-3}{2}   \\ 0 & 0 & 1  } \\
&\xleftrightarrow[]{R_2=R_2+\frac{3}{2}R_3}\myvec{1 & 0 &  0 \\ 0 & 1 & 0  \\ 0 & 0 & 1  } \label{2}
\end{align}
\eqref{2} is the row reduced echelon form of $\vec{A}$ and since it is identity matrix of order 3, we say that vectors $\vec{\alpha_1}$, $\vec{\alpha_2}$, and $\vec{\alpha_3}$  are linearly independent and their column space is $\R^3$ which means vectors $\vec{\alpha_1}$, $\vec{\alpha_2}$, and $\vec{\alpha_3}$  span $\R^3$.
Hence, vectors $\vec{\alpha_1}$, $\vec{\alpha_2}$, and $\vec{\alpha_3}$ form a basis for $\R^3$. \\\\
Now,use theorem \eqref{1}, and calculate the inverse of \eqref{eqA}
then the columns of $\vec{A^{-1}}$ will give the coefficients to write the standard basis vectors in terms of $\alpha_i's$. We try to find the inverse of $\vec{A}$ by row-reducing the augumented matrix.$\vec{A|I}$ 
\begin{align}
	\vec{A} = \myvec{1 & 1 &  0 \\ 0 & 2 & -3  \\ -1 & 1 & 2 }
\end{align}
Now, by row reducing $\vec{A|I}$ as follows 
\begin{multline}
 \myvec{1 & 1 & 0 & 1& 0 &0  \\ 0 & 2 & -3 & 0& 1&0 \\ -1 & 1 & 2& 0& 0 &1} &\xleftrightarrow[]{R_3=R_3+R_1}\myvec{1 & 1 & 0 & 1& 0 &0  \\ 0 & 2 & -3 & 0& 1&0 \\ 0 & 2 & 2& 1& 0 &1}
\end{multline}
\begin{multline}
&\xleftrightarrow[]{R_3=R_3-R_2}\myvec{1 & 1 & 0 & 1& 0 &0  \\ 0 & 2 & -3 & 0& 1&0 \\ 0 & 0 & 5& 1& -1 &1}
\end{multline}
\begin{multline}
&\xleftrightarrow[]{R_2=\frac{R_2}{2}}\myvec{1 & 1 & 0 & 1& 0 &0  \\ 0 & 1 & \frac{-3}{2} & 0& \frac{1}{2}&0 \\ 0 & 0 & 5& 1& -1 &1}
\end{multline}
\begin{multline}
&\xleftrightarrow[]{R_1=R_1-R_2}\myvec{1 & 0 & \frac{3}{2} & 1& \frac{-1}{2} &0  \\ 0 & 1 & \frac{-3}{2} & 0& \frac{1}{2}&0 \\ 0 & 0 & 5& 1& -1 &1}
\end{multline}
\begin{multline}
&\xleftrightarrow[]{R_3=\frac{R_3}{5}}\myvec{1 & 0 & \frac{3}{2} & 1& \frac{-1}{2} &0  \\ 0 & 1 & \frac{-3}{2} & 0& \frac{1}{2}&0 \\ 0 & 0 & 1& \frac{1}{5}& \frac{-1}{5} &\frac{1}{5}}
\end{multline}
\begin{multline}
&\xleftrightarrow[]{R_1=R_1-\frac{3R_3}{2}}\myvec{1 & 0 & 0 & \frac{7}{10}& \frac{-1}{5} &\frac{-3}{10} \\ 0 & 1 & \frac{-3}{2} & 0& \frac{1}{2}&0 \\ 0 & 0 & 1& \frac{1}{5}& \frac{-1}{5} &\frac{1}{5}}
\end{multline}
\begin{multline}
&\xleftrightarrow[]{R_2=R_2+\frac{3R_3}{2}}\myvec{1 & 0 & 0 & \frac{7}{10}& \frac{-1}{5} &\frac{-3}{10} \\ 0 & 1 & 0 & \frac{3}{10}& \frac{1}{5}&\frac{3}{10} \\ 0 & 0 & 1& \frac{1}{5}& \frac{-1}{5} &\frac{1}{5}}	\label{3}
\end{multline}
Thus, by \eqref{3}, we have
\begin{align}
	\vec{A^{-1}}= \myvec{ \frac{7}{10}& \frac{-1}{5} &\frac{-3}{10} \\ \frac{3}{10}& \frac{1}{5}&\frac{3}{10} \\  \frac{1}{5}& \frac{-1}{5} &\frac{1}{5}}
\end{align}
Now, let $\vec{e_1} = \myvec{1 & 0 & 0 }$, $\vec{e_2} = \myvec{0 & 1 & 0 }$, and $\vec{e_3} = \myvec{0 & 0 & 1 }$ be the standard basis for $\R^3$. Hence,each of the standard basis vectors as linear combinations of $\alpha_1, \alpha_2, \alpha_3$ is as under

\begin{align}
& \vec{e_1} = \frac{7}{10}\alpha_1 +\frac{3}{10} \alpha_2+\frac{1}{5}\alpha_3\\ 
& \vec{e_2} = -\frac{1}{5}\alpha_1 +\frac{1}{5} \alpha_2 -\frac{1}{5} \alpha_3 \\
& \vec{e_3} = \frac{-3}{10}\alpha_1 + \frac{3}{10} \alpha_2+\frac{1}{5}\alpha_3 
\end{align}
\end{document}