\documentclass[journal,12pt,twocolumn]{IEEEtran}
%
\usepackage{setspace}
\usepackage{gensymb}
%\doublespacing
\singlespacing

%\usepackage{graphicx}
%\usepackage{amssymb}
%\usepackage{relsize}
\usepackage[cmex10]{amsmath}
%\usepackage{amsthm}
%\interdisplaylinepenalty=2500
%\savesymbol{iint}
%\usepackage{txfonts}
%\restoresymbol{TXF}{iint}
%\usepackage{wasysym}
\usepackage{amsthm}
%\usepackage{iithtlc}
\usepackage{mathrsfs}
\usepackage{txfonts}
\usepackage{stfloats}
\usepackage{bm}
\usepackage{cite}
\usepackage{cases}
\usepackage{subfig}
%\usepackage{xtab}
\usepackage{longtable}
\usepackage{multirow}
%\usepackage{algorithm}
%\usepackage{algpseudocode}
\usepackage{enumitem}
\usepackage{mathtools}
\usepackage{steinmetz}
\usepackage{tikz}
\usepackage{circuitikz}
\usepackage{verbatim}
\usepackage{tfrupee}
\usepackage[breaklinks=true]{hyperref}
%\usepackage{stmaryrd}
\usepackage{tkz-euclide} % loads  TikZ and tkz-base
%\usetkzobj{all}
\usetikzlibrary{calc,math}
\usepackage{listings}
    \usepackage{color}                                            %%
    \usepackage{array}                                            %%
    \usepackage{longtable}                                        %%
    \usepackage{calc}                                             %%
    \usepackage{multirow}                                         %%
    \usepackage{hhline}                                           %%
    \usepackage{ifthen}                                           %%
  %optionally (for landscape tables embedded in another document): %%
    \usepackage{lscape}     
\usepackage{multicol}
\usepackage{chngcntr}
%\usepackage{enumerate}

%\usepackage{wasysym}
%\newcounter{MYtempeqncnt}
\DeclareMathOperator*{\Res}{Res}
%\renewcommand{\baselinestretch}{2}
\renewcommand\thesection{\arabic{section}}
\renewcommand\thesubsection{\thesection.\arabic{subsection}}
\renewcommand\thesubsubsection{\thesubsection.\arabic{subsubsection}}

\renewcommand\thesectiondis{\arabic{section}}
\renewcommand\thesubsectiondis{\thesectiondis.\arabic{subsection}}
\renewcommand\thesubsubsectiondis{\thesubsectiondis.\arabic{subsubsection}}

% correct bad hyphenation here
\hyphenation{op-tical net-works semi-conduc-tor}
\def\inputGnumericTable{}                                 %%

\lstset{
%language=C,
frame=single, 
breaklines=true,
columns=fullflexible
}
%\lstset{
%language=tex,
%frame=single, 
%breaklines=true
%}

\begin{document}
%
\newtheorem{theorem}{Theorem}[section]
\newtheorem{problem}{Problem}
\newtheorem{proposition}{Proposition}[section]
\newtheorem{lemma}{Lemma}[section]
\newtheorem{corollary}[theorem]{Corollary}
\newtheorem{example}{Example}[section]
\newtheorem{definition}[problem]{Definition}
%\newtheorem{thm}{Theorem}[section] 
%\newtheorem{defn}[thm]{Definition}
%\newtheorem{algorithm}{Algorithm}[section]
%\newtheorem{cor}{Corollary}
\newcommand{\BEQA}{\begin{eqnarray}}
\newcommand{\EEQA}{\end{eqnarray}}
\newcommand{\define}{\stackrel{\triangle}{=}}
\newcommand{\R}{\mathbb{R}}
\bibliographystyle{IEEEtran}
%\bibliographystyle{ieeetr}
\providecommand{\mbf}{\mathbf}
\providecommand{\pr}[1]{\ensuremath{\Pr\left(#1\right)}}
\providecommand{\qfunc}[1]{\ensuremath{Q\left(#1\right)}}
\providecommand{\sbrak}[1]{\ensuremath{{}\left[#1\right]}}
\providecommand{\lsbrak}[1]{\ensuremath{{}\left[#1\right.}}
\providecommand{\rsbrak}[1]{\ensuremath{{}\left.#1\right]}}
\providecommand{\brak}[1]{\ensuremath{\left(#1\right)}}
\providecommand{\lbrak}[1]{\ensuremath{\left(#1\right.}}
\providecommand{\rbrak}[1]{\ensuremath{\left.#1\right)}}
\providecommand{\cbrak}[1]{\ensuremath{\left\{#1\right\}}}
\providecommand{\lcbrak}[1]{\ensuremath{\left\{#1\right.}}
\providecommand{\rcbrak}[1]{\ensuremath{\left.#1\right\}}}
\theoremstyle{remark}
\newtheorem{rem}{Remark}
\newcommand{\sgn}{\mathop{\mathrm{sgn}}}
\providecommand{\abs}[1]{\ensuremath{\left\vert#1\right\vert}}
\providecommand{\res}[1]{\Res\displaylimits_{#1}} 
\providecommand{\norm}[1]{\ensuremath{\left\lVert#1\right\rVert}}
%\providecommand{\norm}[1]{\lVert#1\rVert}
\providecommand{\mtx}[1]{\mathbf{#1}}
\providecommand{\mean}[1]{\ensuremath{E\left[ #1 \right]}}
\providecommand{\fourier}{\overset{\mathcal{F}}{ \rightleftharpoons}}
%\providecommand{\hilbert}{\overset{\mathcal{H}}{ \rightleftharpoons}}
\providecommand{\system}{\overset{\mathcal{H}}{ \longleftrightarrow}}
	%\newcommand{\solution}[2]{\textbf{Solution:}{#1}}
\newcommand{\solution}{\noindent \textbf{Solution: }}
\newcommand{\cosec}{\,\text{cosec}\,}
\providecommand{\dec}[2]{\ensuremath{\overset{#1}{\underset{#2}{\gtrless}}}}
\newcommand{\myvec}[1]{\ensuremath{\begin{pmatrix}#1\end{pmatrix}}}
\newcommand{\mydet}[1]{\ensuremath{\begin{vmatrix}#1\end{vmatrix}}}
\newcommand\inv[1]{#1\raisebox{1.15ex}{$\scriptscriptstyle-\!1$}}
%\numberwithin{equation}{section}
\numberwithin{equation}{subsection}
%\numberwithin{problem}{section}
%\numberwithin{definition}{section}
\makeatletter
\@addtoreset{figure}{problem}
\makeatother
\let\StandardTheFigure\thefigure
\let\vec\mathbf
%\renewcommand{\thefigure}{\theproblem.\arabic{figure}}
\renewcommand{\thefigure}{\theproblem}
%\setlist[enumerate,1]{before=\renewcommand\theequation{\theenumi.\arabic{equation}}
%\counterwithin{equation}{enumi}
%\renewcommand{\theequation}{\arabic{subsection}.\arabic{equation}}
\def\putbox#1#2#3{\makebox[0in][l]{\makebox[#1][l]{}\raisebox{\baselineskip}[0in][0in]{\raisebox{#2}[0in][0in]{#3}}}}
     \def\rightbox#1{\makebox[0in][r]{#1}}
     \def\centbox#1{\makebox[0in]{#1}}
     \def\topbox#1{\raisebox{-\baselineskip}[0in][0in]{#1}}
     \def\midbox#1{\raisebox{-0.5\baselineskip}[0in][0in]{#1}}
\vspace{3cm}
\title{EE5609: Matrix Theory\\
          Assignment-10\\}
\author{Major Saurabh Joshi\\MTech Artificial Intelligence\\AI20MTECH13002 }
\maketitle
\newpage
%\tableofcontents
\bigskip
\renewcommand{\thefigure}{\theenumi}
\renewcommand{\thetable}{\theenumi}
\begin{abstract}
This document solves problem on linear independence .  
\end{abstract}
Download the latex code from 
%
%
%
\begin{lstlisting}
https://github.com/saurabh13002/EE5609/tree/master/Assignment10
\end{lstlisting}
%
\section{Problem}
Let $\vec{V}$ be the vector space over the complex numbers of all functions from $\mathbb{R}$ into $\mathbb{C}$, i.e., the space of all complex-valued functions on the real line. Let $f_1(x)=1$, $f_2(x)=e^{ix}$, $f_3(x)=e^{-ix}$.\\\\
(a) Prove that $f_1$, $f_2$, and $f_3$ are linearly independent.
(b) Let $g_1(x)=1$, $g_2(x)=\cos{x}$, $g_3(x)=\sin{x}$. Find an invertible $3\times3$ matrix $\vec{P}$ such that
\begin{align}
    g_j=\sum\limits_{i=1}^3 \vec{P}_{ij}f_i\label{eq}
\end{align}
\section{Solution}
\subsection{a}
Given,
\begin{align}
    f_1(x)=1\label{f1}\\
    f_2(x)=e^{ix}\label{f2}\\
    f_3(x)=e^{-ix}\label{f3}
\end{align}
For $f_1$, $f_2$, and $f_3$ to be linearly independent, the following condition must satisfy.
\begin{align}
    \alpha_1+\alpha_2 f_2+\alpha_3 f_3=0\label{indeq}
\end{align}
$\forall \alpha_i=0$ and $i=$1,2,3
Substitute \eqref{f1},\eqref{f2}, \eqref{f3} in \eqref{indeq}, we get
\begin{align}
    \alpha_1+\alpha_2e^{ix}+\alpha_3e^{-ix}=0\label{eq1}
\end{align}
By Eulers Formula in \eqref{eq1} i.e
\begin{align} 
&  \ e^{ix}=\cos{x}+i\sin{x} \\
&\alpha_1+\alpha_2\cos{x}+i \alpha_2\sin{x}+\alpha_3\cos{x}-i \alpha_3\sin{x}=0
\end{align}
equating real and imaginary parts
\begin{align}
&\alpha_1+\alpha_2\cos{x}+\alpha_3\cos{x}=0\label{eq2}\\
&\alpha_2\sin{x} -\alpha_3\sin{x}=0\label{eq3}
\end{align}
Therefore from \eqref{eq3}
\begin{align}
&\alpha_2=\alpha_3\label{eq5}
\end{align}
substituting \eqref{eq5} in \eqref{eq2}
\begin{align}
&\alpha_1+\alpha_2\cos{x}+\alpha_3\cos{x}=0\label{eq6}\\
&\implies \alpha_1+2\alpha_3\cos{x}=0\label{eq4}
\end{align}
differentiating \eqref{eq4} wrt x
\begin{align}
&-2\alpha_3\sin{x}=0\end{align}
Therefore
\begin{align}
&\alpha_3=0\end{align}
and from \eqref{eq5} 
\begin{align}
    &\alpha_3=\alpha_2=0\label{eq9}
\end{align}
Substituting, \eqref{eq9} in \eqref{eq6} 
\begin{align}
&\alpha_1=0    
\end{align}
Therefore $\alpha_1=0$.
 Thus, from \eqref{eq1}
 \begin{align}
&\alpha_1+\alpha_2 e^{ix}+\alpha_3 e^{-ix}=0 \\
 &\implies \alpha_1=\alpha_2=\alpha_3=0    
 \end{align} and
 \begin{align}
& \alpha_1+\alpha_2 e^{ix}+\alpha_3 e^{-ix}=0 \iff &\alpha_1=\alpha_2=\alpha_3=0    
 \end{align}
 Hence, $f_1, f_2, and f_3$ are linearly independent
\subsection{b}
Given,
\begin{align}
    g_1(x)&=1=f_1\label{g1}\\
    g_2(x)&=\cos{x}=\frac{e^{ix}+e{-ix}}{2}=\frac{f_2}{2}+\frac{f_3}{2}\label{g2}\\
    g_3(x)&=\sin{x}=\frac{e^{ix}-e{-ix}}{2i}=\frac{f_2}{2i}-\frac{f_3}{2i}=-\frac{i}{2}f_2+\frac{i}{2}f_3\label{g3}
\end{align}
Now \eqref{g1}, \eqref{g2}, \eqref{g3} can be converted to matrix form as below.
\begin{align}
    \myvec{g_1&g_2&g_3}=\myvec{f_1&f_2&f_3}\myvec{1&0&0\\0&\frac{1}{2}&-\frac{i}{2}\\0&\frac{1}{2}&\frac{i}{2}}
\end{align}
Therefore, on comparing with \eqref{eq} we get
\begin{align}
    \vec{P}=\myvec{1&0&0\\0&\frac{1}{2}&-\frac{i}{2}\\0&\frac{1}{2}&\frac{i}{2}}
\end{align}
Now to verify invertibility  of $\Vec{P}$ we use row reduction.
\begin{align}
    \myvec{1&0&0\\0&\frac{1}{2}&-\frac{i}{2}\\0&\frac{1}{2}&\frac{i}{2}}\xleftrightarrow{R_3=R_3-R_2}\myvec{1&0&0\\0&\frac{1}{2}&-\frac{i}{2}\\0&0&i}
\end{align}
we got rank of matrix $\vec{P}$ = 3 and is a full rank matrix. Therefore, $\vec{P}$ is invertible.
Hence verified.\\\\



 \end{document}






\begin{align}
     \alpha_1+\alpha_2 f_2+\alpha_3 f_3=0\label{indeq}
      \alpha_1+\alpha_2 e^{ix}+\alpha_3 e^{-ix}=0\label{eq1}
      By Eulers Formula 
      e^{ix}=\cos{x}+i\sin{x}
      \implies \alpha_1+ \cos{x}+i\alpha_2\sin{x}+\alpha_3\cos{x}-i\alpha_3\sin{x}=0
      equating real and imaginary parts
      \alpha_1+\alpha_2\cos{x}+\alpha_3\cos{x}=0\label{indeq2}
      and
      \alpha_2\sin{x} -\alpha_3\sin{x}=0\label{indeq3}
      \therefore \alpha_2=\alpha_3\label{indeq5}
      putting \alpha_2=\alpha_3 in\eqref{indeq2}
      \alpha_1+\alpha_2\cos{x}+\alpha_3\cos{x}=0\label{indeq6}
      \implies \alpha_1+2\alpha_3\cos{x}=0\label{indeq4}
      differentiating\eqref{indeq4} wrt x
      -2\alpha_3\sin{x}=0
      Therefore \alpha_3=0 and from\eqref{indeq5} \alpha_3=\alpha_2=0
      putting, \alpha_3=0 in\eqref{indeq6} 
      Therefore \alpha_1=0
      Thus,from\eqref{eq1} \alpha_1+\alpha_2 e^{ix}+\alpha_3 e^{-ix}=0 
      \implies \alpha_1=\alpha_2=\alpha_3=0 and
      Therefore,
       \alpha_1+\alpha_2 e^{ix}+\alpha_3 e^{-ix}=0 \iff \alpha_1=\alpha_2=\alpha_3=0
 Hence, f_1, f_2, and f_3 are linearly independent.
      
      
      
      
\end{align}