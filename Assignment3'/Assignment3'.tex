\documentclass[journal,12pt,twocolumn]{IEEEtran}
%
\usepackage{setspace}
\usepackage{gensymb}
%\doublespacing
\singlespacing

%\usepackage{graphicx}
%\usepackage{amssymb}
%\usepackage{relsize}
\usepackage[cmex10]{amsmath}
%\usepackage{amsthm}
%\interdisplaylinepenalty=2500
%\savesymbol{iint}
%\usepackage{txfonts}
%\restoresymbol{TXF}{iint}
%\usepackage{wasysym}
\usepackage{amsthm}
%\usepackage{iithtlc}
\usepackage{mathrsfs}
\usepackage{txfonts}
\usepackage{stfloats}
\usepackage{bm}
\usepackage{cite}
\usepackage{cases}
\usepackage{subfig}
%\usepackage{xtab}
\usepackage{longtable}
\usepackage{multirow}
%\usepackage{algorithm}
%\usepackage{algpseudocode}
\usepackage[utf8]{inputenc}
\usepackage{tikz}
\usetikzlibrary{positioning}
\usepackage{enumitem}
\usepackage{mathtools}
\usepackage{steinmetz}
\usepackage{tikz}
\usepackage{circuitikz}
\usepackage{verbatim}
\usepackage{tfrupee}
\usepackage[breaklinks=true]{hyperref}
%\usepackage{stmaryrd}
\usepackage{tkz-euclide} % loads  TikZ and tkz-base
%\usetkzobj{all}
\usetikzlibrary{calc,math}
\usepackage{listings}
    \usepackage{color}                                            %%
    \usepackage{array}                                            %%
    \usepackage{longtable}                                        %%
    \usepackage{calc}                                             %%
    \usepackage{multirow}                                         %%
    \usepackage{hhline}                                           %%
    \usepackage{ifthen}                                           %%
  %optionally (for landscape tables embedded in another document): %%
    \usepackage{lscape}     
\usepackage{multicol}
\usepackage{chngcntr}
%\usepackage{enumerate}

%\usepackage{wasysym}
%\newcounter{MYtempeqncnt}
\DeclareMathOperator*{\Res}{Res}
%\renewcommand{\baselinestretch}{2}
\renewcommand\thesection{\arabic{section}}
\renewcommand\thesubsection{\thesection.\arabic{subsection}}
\renewcommand\thesubsubsection{\thesubsection.\arabic{subsubsection}}

\renewcommand\thesectiondis{\arabic{section}}
\renewcommand\thesubsectiondis{\thesectiondis.\arabic{subsection}}
\renewcommand\thesubsubsectiondis{\thesubsectiondis.\arabic{subsubsection}}

% correct bad hyphenation here
\hyphenation{op-tical net-works semi-conduc-tor}
\def\inputGnumericTable{}                                 %%

\lstset{
%language=C,
frame=single, 
breaklines=true,
columns=fullflexible
}
%\lstset{
%language=tex,
%frame=single, 
%breaklines=true
%}

\begin{document}
%


\newtheorem{theorem}{Theorem}[section]
\newtheorem{problem}{Problem}
\newtheorem{proposition}{Proposition}[section]
\newtheorem{lemma}{Lemma}[section]
\newtheorem{corollary}[theorem]{Corollary}
\newtheorem{example}{Example}[section]
\newtheorem{definition}[problem]{Definition}
%\newtheorem{thm}{Theorem}[section] 
%\newtheorem{defn}[thm]{Definition}
%\newtheorem{algorithm}{Algorithm}[section]
%\newtheorem{cor}{Corollary}
\newcommand{\BEQA}{\begin{eqnarray}}
\newcommand{\EEQA}{\end{eqnarray}}
\newcommand{\define}{\stackrel{\triangle}{=}}
\bibliographystyle{IEEEtran}
\providecommand{\mbf}{\mathbf}
\providecommand{\pr}[1]{\ensuremath{\Pr\left(#1\right)}}
\providecommand{\qfunc}[1]{\ensuremath{Q\left(#1\right)}}
\providecommand{\sbrak}[1]{\ensuremath{{}\left[#1\right]}}
\providecommand{\lsbrak}[1]{\ensuremath{{}\left[#1\right.}}
\providecommand{\rsbrak}[1]{\ensuremath{{}\left.#1\right]}}
\providecommand{\brak}[1]{\ensuremath{\left(#1\right)}}
\providecommand{\lbrak}[1]{\ensuremath{\left(#1\right.}}
\providecommand{\rbrak}[1]{\ensuremath{\left.#1\right)}}
\providecommand{\cbrak}[1]{\ensuremath{\left\{#1\right\}}}
\providecommand{\lcbrak}[1]{\ensuremath{\left\{#1\right.}}
\providecommand{\rcbrak}[1]{\ensuremath{\left.#1\right\}}}
\theoremstyle{remark}
\newtheorem{rem}{Remark}
\newcommand{\sgn}{\mathop{\mathrm{sgn}}}
\providecommand{\abs}[1]{\ensuremath{\left\vert#1\right\vert}}
\providecommand{\res}[1]{\Res\displaylimits_{#1}} 
\providecommand{\norm}[1]{\ensuremath{\left\lVert#1\right\rVert}}
%\providecommand{\norm}[1]{\lVert#1\rVert}
\providecommand{\mtx}[1]{\mathbf{#1}}
\providecommand{\mean}[1]{\ensuremath{E\left[ #1 \right]}}
\providecommand{\fourier}{\overset{\mathcal{F}}{ \rightleftharpoons}}
%\providecommand{\hilbert}{\overset{\mathcal{H}}{ \rightleftharpoons}}
\providecommand{\system}{\overset{\mathcal{H}}{ \longleftrightarrow}}
	%\newcommand{\solution}[2]{\textbf{Solution:}{#1}}
\newcommand{\solution}{\noindent \textbf{Solution: }}
\newcommand{\cosec}{\,\text{cosec}\,}
\providecommand{\dec}[2]{\ensuremath{\overset{#1}{\underset{#2}{\gtrless}}}}
\newcommand{\myvec}[1]{\ensuremath{\begin{pmatrix}#1\end{pmatrix}}}
\newcommand{\mydet}[1]{\ensuremath{\begin{vmatrix}#1\end{vmatrix}}}
%\numberwithin{equation}{section}
\numberwithin{equation}{subsection}
%\numberwithin{problem}{section}
%\numberwithin{definition}{section}
\makeatletter
\@addtoreset{figure}{problem}
\makeatother
\let\StandardTheFigure\thefigure
\let\vec\mathbf
%\renewcommand{\thefigure}{\theproblem.\arabic{figure}}
\renewcommand{\thefigure}{\theproblem}
%\setlist[enumerate,1]{before=\renewcommand\theequation{\theenumi.\arabic{equation}}
%\counterwithin{equation}{enumi}
%\renewcommand{\theequation}{\arabic{subsection}.\arabic{equation}}
\def\putbox#1#2#3{\makebox[0in][l]{\makebox[#1][l]{}\raisebox{\baselineskip}[0in][0in]{\raisebox{#2}[0in][0in]{#3}}}}
     \def\rightbox#1{\makebox[0in][r]{#1}}
     \def\centbox#1{\makebox[0in]{#1}}
     \def\topbox#1{\raisebox{-\baselineskip}[0in][0in]{#1}}
     \def\midbox#1{\raisebox{-0.5\baselineskip}[0in][0in]{#1}}
\vspace{3cm}
\title{Matrix Theory (EE5609) Assignment 3}
\author{Major Saurabh Joshi\\MTech Artificial Intelligence\\AI20MTECH13002}
\maketitle
\newpage
%\tableofcontents
\bigskip
\renewcommand{\thefigure}{\theenumi}
\renewcommand{\thetable}{\theenumi}
\begin{abstract}
This  document provides solution to problem 1.42 of Triangle Exercises.
\end{abstract}
Download all latex-tikz codes from 
%
\begin{lstlisting}
https://github.com/ Saurabh 13002/EE5609/tree/master
\end{lstlisting}
%
\section{Problem}
Sides AB and AC of $\triangle{ABC}$. ABC are extended to points P and Q respectively.
Also, $\angle PBC < \angle QCB$. Show that $AC > AB$.
\section{ Solution}
\renewcommand{\thefigure}{1}
\begin{figure}[hb]
	\centering
	\centering
	\resizebox{\columnwidth}{!}{
}
	\caption{\triangle ABC}
	\end{figure}
	
Given conditions are : $\triangle {ABC}$  is a triangle having \angle PBC < \angle QCB 

\begin{align}
\frac{\left( \vec{P - B} \right)^T  \left( \vec{B - C } \right)}{\norm{\vec{ P - B}} \norm{\vec{B - C}}}<\frac{\left( \vec{ Q - C} \right)^T  \left( \vec{C - B} \right)}{\norm{\vec{ Q - C}} \norm{\vec{C - B}}}
\end{align}
\begin{multline}\implies\frac{\left( \vec{P - B+A-B} \right)^T  \left( \vec{B - C } \right)}{\norm{\vec{ P - B}} }<\\\frac{\left( \vec{ Q - C+A-C} \right)^T  \left( \vec{C - B} \right)}{\norm{\vec{ Q - C}}}
\end{multline}
As,$\vec{ P - B}$,$\vec{ A - B}$ and $\vec{ Q - C}$,$\vec{ A - C}$ are in same direction.

\begin{multline}
\implies\frac{\left( \vec{A - B+A-B} \right)^T  \left( \vec{B - C } \right)}{\norm{\vec{ A - B}} }<\\\frac{\left( \vec{ A - C+A-C} \right)^T  \left( \vec{C - B} \right)}{\norm{\vec{ A - C}}}
\end{multline}

\begin{align}
\implies\frac{\left( \vec{A - B} \right)^T  \left( \vec{B - C } \right)}{\norm{\vec{ A - B}} }<\frac{\left( \vec{ A - C} \right)^T  \left( \vec{C - B} \right)}{\norm{\vec{ A - C}}}
\end{align}
On, rewriting the equation above;
\begin{multline}
\implies\frac{\norm{\vec{ A - B}}^2 - \left( \vec{A - B } \right)^T \left( \vec{A - C } \right)}{\norm{\vec{ A - B}}}<\\ \frac{\norm{\vec{ A - C}}^2 - \left( \vec{A - C } \right)^T \left( \vec{A - B } \right)}{\norm{\vec{ A - C}}}
\end{multline}
Upon rearranging the terms, we have
\begin{multline}
\implies\norm{\vec{ A - B}}-\frac{ \left( \vec{A - B } \right)^T \left( \vec{A - C } \right)}{\norm{\vec{ A - B}}}<\\ \norm{\vec{ A - C}}-\frac{ \left( \vec{A - C } \right)^T \left( \vec{A - B } \right)}{\norm{\vec{ A - C}}}\label{eq1}
\end{multline}
We know that 
\begin{align}
\cos\angle BAC= \frac{\left( \vec{A - B } \right)^T \left( \vec{A - C } \right)}{\norm{\vec{ A - B}}\norm{\vec{ A - C}}}
\end{align}
Re-writing \eqref{eq1}
\begin{multline}
\norm{\vec{A - B} }-\norm{\vec{A - C} }\cos\angle BAC<\\\norm{\vec{ A - C}}-\norm{\vec{A - B} }\cos\angle BAC
\end{multline}  
Upon rearranging,
\begin{multline}
\implies\norm{\vec{A - B} }+\norm{\vec{A - B} }\cos\angle BAC<\\\norm{\vec{A - C}}+\norm{\vec{A - C} }\cos\angle BAC\end{multline}
\begin{multline}
\implies
\norm{\vec{A - B} }(1+\cos\angle BAC)<\\\norm{\vec{A - C} }(1+\cos\angle BAC)\end{multline}
Therefore,
$\norm{\vec{A - B}} < \norm{\vec{A - C}}$\\
Hence Proved

\end{document}
\\

