\documentclass[journal,12pt,onecolumn]{IEEEtran}

\usepackage{setspace}
\usepackage{gensymb}
\singlespacing


\usepackage[cmex10]{amsmath}

\usepackage{amsthm}

\usepackage{mathrsfs}
\usepackage{txfonts}
\usepackage{stfloats}
\usepackage{bm}
\usepackage{cite}
\usepackage{cases}
\usepackage{subfig}

\usepackage{longtable}
\usepackage{multirow}

\usepackage{enumitem}
\usepackage{mathtools}
\usepackage{steinmetz}
\usepackage{tikz}
\usepackage{circuitikz}
\usepackage{verbatim}
\usepackage{tfrupee}
\usepackage[breaklinks=true]{hyperref}
\usepackage{graphicx}
\usepackage{tkz-euclide}

\usetikzlibrary{calc,math}
\usepackage{listings}
    \usepackage{color}                                            %%
    \usepackage{array}                                            %%
    \usepackage{longtable}                                        %%
    \usepackage{calc}                                             %%
    \usepackage{multirow}                                         %%
    \usepackage{hhline}                                           %%
    \usepackage{ifthen}                                           %%
    \usepackage{lscape}     
\usepackage{multicol}
\usepackage{chngcntr}

\DeclareMathOperator*{\Res}{Res}

\renewcommand\thesection{\arabic{section}}
\renewcommand\thesubsection{\thesection.\arabic{subsection}}
\renewcommand\thesubsubsection{\thesubsection.\arabic{subsubsection}}

\renewcommand\thesectiondis{\arabic{section}}
\renewcommand\thesubsectiondis{\thesectiondis.\arabic{subsection}}
\renewcommand\thesubsubsectiondis{\thesubsectiondis.\arabic{subsubsection}}


\hyphenation{op-tical net-works semi-conduc-tor}
\def\inputGnumericTable{}                                 %%

\lstset{
%language=C,
frame=single, 
breaklines=true,
columns=fullflexible
}
\begin{document}


\newtheorem{theorem}{Theorem}[section]
\newtheorem{problem}{Problem}
\newtheorem{proposition}{Proposition}[section]
\newtheorem{lemma}{Lemma}[section]
\newtheorem{corollary}[theorem]{Corollary}
\newtheorem{example}{Example}[section]
\newtheorem{definition}[problem]{Definition}

\newcommand{\BEQA}{\begin{eqnarray}}
\newcommand{\EEQA}{\end{eqnarray}}
\newcommand{\define}{\stackrel{\triangle}{=}}
\bibliographystyle{IEEEtran}
\providecommand{\mbf}{\mathbf}
\providecommand{\pr}[1]{\ensuremath{\Pr\left(#1\right)}}
\providecommand{\qfunc}[1]{\ensuremath{Q\left(#1\right)}}
\providecommand{\sbrak}[1]{\ensuremath{{}\left[#1\right]}}
\providecommand{\lsbrak}[1]{\ensuremath{{}\left[#1\right.}}
\providecommand{\rsbrak}[1]{\ensuremath{{}\left.#1\right]}}
\providecommand{\brak}[1]{\ensuremath{\left(#1\right)}}
\providecommand{\lbrak}[1]{\ensuremath{\left(#1\right.}}
\providecommand{\rbrak}[1]{\ensuremath{\left.#1\right)}}
\providecommand{\cbrak}[1]{\ensuremath{\left\{#1\right\}}}
\providecommand{\lcbrak}[1]{\ensuremath{\left\{#1\right.}}
\providecommand{\rcbrak}[1]{\ensuremath{\left.#1\right\}}}
\theoremstyle{remark}
\newtheorem{rem}{Remark}
\newcommand{\sgn}{\mathop{\mathrm{sgn}}}
\providecommand{\abs}[1]{\ensuremath{\left\vert#1\right\vert}}
\providecommand{\res}[1]{\Res\displaylimits_{#1}} 
\providecommand{\norm}[1]{\ensuremath{\left\lVert#1\right\rVert}}
%\providecommand{\norm}[1]{\lVert#1\rVert}
\providecommand{\mtx}[1]{\mathbf{#1}}
\providecommand{\mean}[1]{E\ensuremath{\left[ #1 \right]}}
\providecommand{\fourier}{\overset{\mathcal{F}}{ \rightleftharpoons}}
%\providecommand{\hilbert}{\overset{\mathcal{H}}{ \rightleftharpoons}}
\providecommand{\system}{\overset{\mathcal{H}}{ \longleftrightarrow}}
	%\newcommand{\solution}[2]{\textbf{Solution:}{#1}}
\newcommand{\solution}{\noindent \textbf{Solution: }}
\newcommand{\cosec}{\,\text{cosec}\,}
\newcommand{\R}{\mathbb{R}}
\providecommand{\dec}[2]{\ensuremath{\overset{#1}{\underset{#2}{\gtrless}}}}
\newcommand{\myvec}[1]{\ensuremath{\begin{pmatrix}#1\end{pmatrix}}}
\newcommand{\mydet}[1]{\ensuremath{\begin{vmatrix}#1\end{vmatrix}}}
\numberwithin{equation}{subsection}
\makeatletter
\@addtoreset{figure}{problem}
\makeatother
\let\StandardTheFigure\thefigure
\let\vec\mathbf
\renewcommand{\thefigure}{\theproblem}
\def\putbox#1#2#3{\makebox[0in][l]{\makebox[#1][l]{}\raisebox{\baselineskip}[0in][0in]{\raisebox{#2}[0in][0in]{#3}}}}
     \def\rightbox#1{\makebox[0in][r]{#1}}
     \def\centbox#1{\makebox[0in]{#1}}
     \def\topbox#1{\raisebox{-\baselineskip}[0in][0in]{#1}}
     \def\midbox#1{\raisebox{-0.5\baselineskip}[0in][0in]{#1}}
\vspace{3cm}
\onecolumn
\title{EE5609: Matrix Theory\\
          Assignment-13\\}
\author{Major Saurabh Joshi\\MTech Artificial Intelligence\\AI20MTECH13002 }
\maketitle
\bigskip
\renewcommand{\thefigure}{\theenumi}
\renewcommand{\thetable}{\theenumi}
Download codes from 
%
\begin{lstlisting}
https://github.com/saurabh13002/EE5609/tree/master/Assignment13
\end{lstlisting}
%
 
\section{Question}
Let $\vec{A}$,$\vec{B}$ be n × n matrices.Which of the following equals trace$(\vec{A}^2 \vec{B}^2)$?\\

\begin{enumerate}
\item (trace$(\vec{A} \vec{B}))^2.$
\item trace$(\vec{A} \vec{B}^2 \vec{A}).$
\item trace$((\vec{A} \vec{B})^2).$
\item trace$(\vec{B} \vec{A} \vec{B} \vec{A}).$
\end{enumerate}
%
\section{Solution}
\begin{longtable}{|p{5cm}|p{13cm}|}
\hline
\textbf{Statement} &\textbf{Solution}\\
\hline
Definition & The trace of an n × n square matrix $\vec{A}$ is defined as:
\begin{displaymath}tr({\vec{A}}) = \sum_{i=1}^n a_{ii} \end{displaymath}

where $a_{ii}$ denotes the entry on the ith row and ith column of $\vec{A}.$ \\
\hline
Properties&
\parbox{12cm}{\begin{align}\text{The properties of the trace}:
tr(c{\vec{A}})=c\;tr({\vec{A}})\\
tr({\vec{A}}^T)=tr({\vec{A}})\\
tr({\vec{A}}+{\vec{B}})=tr({\vec{B}}+{\vec{A}})\\
tr({\vec{A}}{\vec{B}})=tr({\vec{B}}{\vec{A}})\label{eq}\\
tr({\vec{A}}^T{\vec{B}})=tr({\vec{A}}{\vec{B}}^T)\\
tr({\vec{R}}^{-1}{\vec{A}}{\vec{R}})=tr({\vec{R}}^{-1}({\vec{A}}{\vec{R}}))\\
=tr(({\vec{A}}{\vec{R}}){\vec{R}}^{-1})=tr({\vec{A}})
\end{align}}\\
\hline
Checking $tr$$(\vec{A}^2 \vec{B}^2).$ &
\parbox{12cm}{\begin{align}
 \text{Upon rewriting and from \eqref{eq}, } tr(\vec{A}^2 \vec{B}^2)=tr(\vec{A}\vec{A} \vec{B}\vec{B})\\
           = tr(\vec{B}\vec{A} \vec{A}\vec{B})\\
           = tr(\vec{B}\vec{B} \vec{A}\vec{A})\\
           = tr(\vec{A}\vec{B} \vec{B}\vec{A})\label{eq1}\\
           = tr(\vec{A}\vec{A} \vec{B}\vec{B})\\
           = tr(\vec{A}^2 \vec{B}^2)\label{eq2}
\end{align}}\\
\hline
Checking ($tr$$(\vec{A} \vec{B}))^2.$&
\parbox{12cm}{\begin{align}\text{from \eqref{eq},}\
    (tr(\vec{A} \vec{B}))^2=(tr(\vec{B} \vec{A}))^2
\end{align}}\\
\hline
 Checking $tr$$(\vec{A} \vec{B}^2 \vec{A}).$&
\parbox{12cm}{\begin{align}\text{Rewriting,}
\  tr(\vec{A} \vec{B}^2 \vec{A}) =tr(\vec{A}\vec{B} \vec{B}\vec{A})\\\text{ from \eqref{eq},}\
 tr(\vec{A} \vec{B}^2 \vec{A})=  tr(\vec{A}\vec{A} \vec{B}\vec{B})\label{eq3}\label{eql}=tr(\vec{A}^2 \vec{B}^2)  
\end{align}}\\
\hline
Checking $tr$$(\vec{A} \vec{B})^2.$&
\parbox{12cm}{\begin{align}\text{from \eqref{eq},}\
 tr(\vec{A} \vec{B})^2=tr(\vec{B}\vec{A})^2 
\end{align}}\\
\hline
 Checking $tr$$(\vec{B} \vec{A} \vec{B} \vec{A}).$&
\parbox{12cm}{\begin{align}\text{from \eqref{eq}}\\
 tr(\vec{B} \vec{A} \vec{B} \vec{A}) =tr(\vec{A}\vec{B} \vec{A}\vec{B})\\
 =tr(\vec{B}\vec{A} \vec{B}\vec{A})
 \end{align}}\\
\hline
 Conclusion&
Hence, from \eqref{eq}, and \eqref{eql} option 2, ie  $tr$$(\vec{A} \vec{B}^2 \vec{A}).$ is the correct answer.\\
\hline
\caption*{Table1:Solution}
\end{longtable}
\end{document}
