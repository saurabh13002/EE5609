\documentclass[journal,12pt,twocolumn]{IEEEtran}

\usepackage{setspace}
\usepackage{gensymb}
\usepackage{supertabular}
\singlespacing


\usepackage[cmex10]{amsmath}

\usepackage{amsthm}

\usepackage{mathrsfs}
\usepackage{txfonts}
\usepackage{stfloats}
\usepackage{bm}
\usepackage{cite}
\usepackage{cases}
\usepackage{subfig}

\usepackage{longtable}
\usepackage{multirow}

\usepackage{enumitem}
\usepackage{mathtools}
\usepackage{steinmetz}
\usepackage{tikz}
\usepackage{circuitikz}
\usepackage{verbatim}
\usepackage{tfrupee}
\usepackage[breaklinks=true]{hyperref}

\usepackage{tkz-euclide}

\usetikzlibrary{calc,math}
\usepackage{listings}
    \usepackage{color}                                            %%
    \usepackage{array}                                            %%
    \usepackage{longtable}                                        %%
    \usepackage{calc}                                             %%
    \usepackage{multirow}                                         %%
    \usepackage{hhline}                                           %%
    \usepackage{ifthen}                                           %%
    \usepackage{lscape}     
\usepackage{multicol}
\usepackage{chngcntr}
\usepackage{graphicx}
\DeclareMathOperator*{\Res}{Res}

\renewcommand\thesection{\arabic{section}}
\renewcommand\thesubsection{\thesection.\arabic{subsection}}
\renewcommand\thesubsubsection{\thesubsection.\arabic{subsubsection}}

\renewcommand\thesectiondis{\arabic{section}}
\renewcommand\thesubsectiondis{\thesectiondis.\arabic{subsection}}
\renewcommand\thesubsubsectiondis{\thesubsectiondis.\arabic{subsubsection}}


\hyphenation{op-tical net-works semi-conduc-tor}
\def\inputGnumericTable{}                                 %%

\lstset{
%language=C,
frame=single, 
breaklines=true,
columns=fullflexible
}
\begin{document}


\newtheorem{theorem}{Theorem}[section]
\newtheorem{problem}{Problem}
\newtheorem{proposition}{Proposition}[section]
\newtheorem{lemma}{Lemma}[section]
\newtheorem{corollary}[theorem]{Corollary}
\newtheorem{example}{Example}[section]
\newtheorem{definition}[problem]{Definition}

\newcommand{\BEQA}{\begin{eqnarray}}
\newcommand{\EEQA}{\end{eqnarray}}
\newcommand{\define}{\stackrel{\triangle}{=}}
\bibliographystyle{IEEEtran}
\providecommand{\mbf}{\mathbf}
\providecommand{\pr}[1]{\ensuremath{\Pr\left(#1\right)}}
\providecommand{\qfunc}[1]{\ensuremath{Q\left(#1\right)}}
\providecommand{\sbrak}[1]{\ensuremath{{}\left[#1\right]}}
\providecommand{\lsbrak}[1]{\ensuremath{{}\left[#1\right.}}
\providecommand{\rsbrak}[1]{\ensuremath{{}\left.#1\right]}}
\providecommand{\brak}[1]{\ensuremath{\left(#1\right)}}
\providecommand{\lbrak}[1]{\ensuremath{\left(#1\right.}}
\providecommand{\rbrak}[1]{\ensuremath{\left.#1\right)}}
\providecommand{\cbrak}[1]{\ensuremath{\left\{#1\right\}}}
\providecommand{\lcbrak}[1]{\ensuremath{\left\{#1\right.}}
\providecommand{\rcbrak}[1]{\ensuremath{\left.#1\right\}}}
\theoremstyle{remark}
\newtheorem{rem}{Remark}
\newcommand{\sgn}{\mathop{\mathrm{sgn}}}
\providecommand{\abs}[1]{\ensuremath{\left\vert#1\right\vert}}
\providecommand{\res}[1]{\Res\displaylimits_{#1}} 
\providecommand{\norm}[1]{\ensuremath{\left\lVert#1\right\rVert}}
%\providecommand{\norm}[1]{\lVert#1\rVert}
\providecommand{\mtx}[1]{\mathbf{#1}}
\providecommand{\mean}[1]{E\ensuremath{\left[ #1 \right]}}
\providecommand{\fourier}{\overset{\mathcal{F}}{ \rightleftharpoons}}
%\providecommand{\hilbert}{\overset{\mathcal{H}}{ \rightleftharpoons}}
\providecommand{\system}{\overset{\mathcal{H}}{ \longleftrightarrow}}
	%\newcommand{\solution}[2]{\textbf{Solution:}{#1}}
\newcommand{\solution}{\noindent \textbf{Solution: }}
\newcommand{\cosec}{\,\text{cosec}\,}
\newcommand{\R}{\mathbb{R}}
\providecommand{\dec}[2]{\ensuremath{\overset{#1}{\underset{#2}{\gtrless}}}}
\newcommand{\myvec}[1]{\ensuremath{\begin{pmatrix}#1\end{pmatrix}}}
\newcommand{\mydet}[1]{\ensuremath{\begin{vmatrix}#1\end{vmatrix}}}
\numberwithin{equation}{subsection}
\makeatletter
\@addtoreset{figure}{problem}
\makeatother
\let\StandardTheFigure\thefigure
\let\vec\mathbf
\renewcommand{\thefigure}{\theproblem}
\def\putbox#1#2#3{\makebox[0in][l]{\makebox[#1][l]{}\raisebox{\baselineskip}[0in][0in]{\raisebox{#2}[0in][0in]{#3}}}}
     \def\rightbox#1{\makebox[0in][r]{#1}}
     \def\centbox#1{\makebox[0in]{#1}}
     \def\topbox#1{\raisebox{-\baselineskip}[0in][0in]{#1}}
     \def\midbox#1{\raisebox{-0.5\baselineskip}[0in][0in]{#1}}
\vspace{3cm}
\onecolumn
\title{EE5609: Matrix Theory\\
          Assignment-11\\}
\author{Major Saurabh Joshi\\MTech Artificial Intelligence\\AI20MTECH13002 }
\maketitle
\bigskip
\renewcommand{\thefigure}{\theenumi}
\renewcommand{\thetable}{\theenumi}
\begin{abstract}
\small\textbf{This document solves problem on Eigen values and properties}.
\end{abstract}
Download all solutions from 
\begin{lstlisting}
https://github.com/saurabh13002/EE5609/tree/master/Assignment11
\end{lstlisting}
\section{\textbf{Problem}}
Let $\vec{A}$ be a real symmetric matrix and  $\vec{B}=\vec{I}+i\vec{A}$, where $i^2=-1$. Then choose the correct option.\\
\begin{enumerate}
	\item  $\vec{B}$ is invertible if and only if $\vec{A}$ is invertible.\\
	\item All Eigenvalues of $\vec{B}$ are necessarily real.\\
	\item $\vec{B}-\vec{I}$ is necessarily invertible.\\
	\item $\vec{B}$ is necessarily invertible.\\
\end{enumerate}
\section{\textbf{Explanation}}
\renewcommand{\thetable}{1}
\begin{longtable}{|l|l|}
\hline
\multirow{3}{*}{} & \\
Statement 1. &$\vec{B}$ is invertible if and only if $\vec{A}$ is  invertible.\\
\hline
& \\
False statement& Matrix $\vec{B}$ is invertible even if $\vec{A}$ is non invertible.\\
\hline
Example:&Consider a matrix \\&\parbox{12cm}{\begin{align}
 \vec{A}=\myvec{
1 &0 \\
0&0}\label{eq1}\end{align}}\\& a real non invertible,symmetric matrix.\\&\parbox{12cm}{\begin{align}
&\implies\vec{B}=\myvec{1 & 0\\ 0 & 1}+i\myvec{1 & 0\\ 0 & 0}=\myvec{1+i & 0\\ 0 & 1  }\label{eq2}\end{align}}\\ 

&is invertible even if $\vec{A}$ is non invertible.\\
\hline

\pagebreak
\hline
\multirow{3}{*}&\\
Statement 2. & All Eigenvalues of $\vec{B}$ are necessarily real.\\
\hline
&\\
False statement& Matrix $\vec{B}$ can have complex Eigenvalues.\\
\hline
Proof  :& Eigen values of $\vec{B}$ = Eigen values of     
($\vec{I}$) + i (Eigen values of $\vec{A}$).
\\&Clearly from \eqref{eq2} above Eigen values of $\vec{B}$ are  $1$ and $1+i$   respectively.\\
& Hence $\vec{B}$ can also have complex Eigen value.\\
     \hline
\multirow{3}{*} & \\ 
Statement 3.  & $\vec{B}-\vec{I}$ is necessarily invertible.\\
\hline
&\\
False statement& $\vec{B}-\vec{I}=i\vec{A}$ will be invertible if $\vec{A}$, is invertible.\\\hline
&\\
Proof: & We have $\vec{B}-\vec{I}=i\vec{A}$\\
&$\implies \vec{B-I}=i\vec{A}=\myvec{i & 0\\ 0 & 0}$,from \eqref{eq1}\\
 &Hence $\vec{B-I}$ is not invertible,unless $\vec{A}$ is invertible. \\
\hline
\multirow{3}{*}&\\
Statement 4. & $\vec{B}$ is necessarily invertible.\\
\hline
& \\
Correct Statement:& Matrix $\vec{B}$ has non zero Eigen values corresponding to Eigenvector $X$ .\\
\hline
Proof:& Let $X$ be an Eigen vector of $\vec{A}$ corresponding to Eigen value $\lambda$\\
&\\
&also,$\lambda\epsilon \R$\\
&\\
&$\implies$ $\vec{A}$$X$=$\lambda X $\\
&\\
&$\therefore$ $\vec{B}$$X=(\vec{I}+i\vec{A})X$= $\vec{I}$$X$+i$\vec{A}$$X$= $X$+$i\lambda$$X$\\
&\\
& $\implies$$\vec{B}$$X$ = $(1+i\lambda)$$X$\\
& Therefore, $1+i\lambda$ is an Eigen value of $\vec{B}$,\\
& corresponding to Eigen vector $X$,which are non zero.\\
& Hence, $\vec{B}$ is necessarily invertible.\\
\hline
\caption{Solution summary}
\label{table:1}
\end{longtable}
\end{document}




