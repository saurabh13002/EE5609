\documentclass[journal,12pt,twocolumn]{IEEEtran}

\usepackage{setspace}
\usepackage{gensymb}
\usepackage{supertabular}
\singlespacing


\usepackage[cmex10]{amsmath}

\usepackage{amsthm}

\usepackage{mathrsfs}
\usepackage{txfonts}
\usepackage{stfloats}
\usepackage{bm}
\usepackage{cite}
\usepackage{cases}
\usepackage{subfig}

\usepackage{longtable}
\usepackage{multirow}

\usepackage{enumitem}
\usepackage{mathtools}
\usepackage{steinmetz}
\usepackage{tikz}
\usepackage{circuitikz}
\usepackage{verbatim}
\usepackage{tfrupee}
\usepackage[breaklinks=true]{hyperref}

\usepackage{tkz-euclide}

\usetikzlibrary{calc,math}
\usepackage{listings}
    \usepackage{color}                                            %%
    \usepackage{array}                                            %%
    \usepackage{longtable}                                        %%
    \usepackage{calc}                                             %%
    \usepackage{multirow}                                         %%
    \usepackage{hhline}                                           %%
    \usepackage{ifthen}                                           %%
    \usepackage{lscape}     
\usepackage{multicol}
\usepackage{chngcntr}
\usepackage{graphicx}
\DeclareMathOperator*{\Res}{Res}

\renewcommand\thesection{\arabic{section}}
\renewcommand\thesubsection{\thesection.\arabic{subsection}}
\renewcommand\thesubsubsection{\thesubsection.\arabic{subsubsection}}

\renewcommand\thesectiondis{\arabic{section}}
\renewcommand\thesubsectiondis{\thesectiondis.\arabic{subsection}}
\renewcommand\thesubsubsectiondis{\thesubsectiondis.\arabic{subsubsection}}


\hyphenation{op-tical net-works semi-conduc-tor}
\def\inputGnumericTable{}                                 %%

\lstset{
%language=C,
frame=single, 
breaklines=true,
columns=fullflexible
}
\begin{document}


\newtheorem{theorem}{Theorem}[section]
\newtheorem{problem}{Problem}
\newtheorem{proposition}{Proposition}[section]
\newtheorem{lemma}{Lemma}[section]
\newtheorem{corollary}[theorem]{Corollary}
\newtheorem{example}{Example}[section]
\newtheorem{definition}[problem]{Definition}

\newcommand{\BEQA}{\begin{eqnarray}}
\newcommand{\EEQA}{\end{eqnarray}}
\newcommand{\define}{\stackrel{\triangle}{=}}
\bibliographystyle{IEEEtran}
\providecommand{\mbf}{\mathbf}
\providecommand{\pr}[1]{\ensuremath{\Pr\left(#1\right)}}
\providecommand{\qfunc}[1]{\ensuremath{Q\left(#1\right)}}
\providecommand{\sbrak}[1]{\ensuremath{{}\left[#1\right]}}
\providecommand{\lsbrak}[1]{\ensuremath{{}\left[#1\right.}}
\providecommand{\rsbrak}[1]{\ensuremath{{}\left.#1\right]}}
\providecommand{\brak}[1]{\ensuremath{\left(#1\right)}}
\providecommand{\lbrak}[1]{\ensuremath{\left(#1\right.}}
\providecommand{\rbrak}[1]{\ensuremath{\left.#1\right)}}
\providecommand{\cbrak}[1]{\ensuremath{\left\{#1\right\}}}
\providecommand{\lcbrak}[1]{\ensuremath{\left\{#1\right.}}
\providecommand{\rcbrak}[1]{\ensuremath{\left.#1\right\}}}
\theoremstyle{remark}
\newtheorem{rem}{Remark}
\newcommand{\sgn}{\mathop{\mathrm{sgn}}}
\providecommand{\abs}[1]{\ensuremath{\left\vert#1\right\vert}}
\providecommand{\res}[1]{\Res\displaylimits_{#1}} 
\providecommand{\norm}[1]{\ensuremath{\left\lVert#1\right\rVert}}
%\providecommand{\norm}[1]{\lVert#1\rVert}
\providecommand{\mtx}[1]{\mathbf{#1}}
\providecommand{\mean}[1]{E\ensuremath{\left[ #1 \right]}}
\providecommand{\fourier}{\overset{\mathcal{F}}{ \rightleftharpoons}}
%\providecommand{\hilbert}{\overset{\mathcal{H}}{ \rightleftharpoons}}
\providecommand{\system}{\overset{\mathcal{H}}{ \longleftrightarrow}}
	%\newcommand{\solution}[2]{\textbf{Solution:}{#1}}
\newcommand{\solution}{\noindent \textbf{Solution: }}
\newcommand{\cosec}{\,\text{cosec}\,}
\newcommand{\R}{\mathbb{R}}
\providecommand{\dec}[2]{\ensuremath{\overset{#1}{\underset{#2}{\gtrless}}}}
\newcommand{\myvec}[1]{\ensuremath{\begin{pmatrix}#1\end{pmatrix}}}
\newcommand{\mydet}[1]{\ensuremath{\begin{vmatrix}#1\end{vmatrix}}}
\numberwithin{equation}{subsection}
\makeatletter
\@addtoreset{figure}{problem}
\makeatother
\let\StandardTheFigure\thefigure
\let\vec\mathbf
\renewcommand{\thefigure}{\theproblem}
\def\putbox#1#2#3{\makebox[0in][l]{\makebox[#1][l]{}\raisebox{\baselineskip}[0in][0in]{\raisebox{#2}[0in][0in]{#3}}}}
     \def\rightbox#1{\makebox[0in][r]{#1}}
     \def\centbox#1{\makebox[0in]{#1}}
     \def\topbox#1{\raisebox{-\baselineskip}[0in][0in]{#1}}
     \def\midbox#1{\raisebox{-0.5\baselineskip}[0in][0in]{#1}}
\vspace{3cm}
\onecolumn
\title{EE5609: Matrix Theory\\
          Assignment-11\\}
\author{Major Saurabh Joshi\\MTech Artificial Intelligence\\AI20MTECH13002 }
\maketitle
\bigskip
\renewcommand{\thefigure}{\theenumi}
\renewcommand{\thetable}{\theenumi}
\begin{abstract}
This document solves problem on Eigen values and properties.
\end{abstract}
Download all solutions from 
\begin{lstlisting}
https://github.com/saurabh13002/EE5609/tree/master/Assignment11
\end{lstlisting}
\section{Problem}
Let $\vec{A}$ be a real symmetric matrix and  $\vec{B}=\vec{I}+i\vec{A}$, where $i^2=-1$. Then\\
1. $\vec{B}$ is invertible if and only if $\vec{A}$ is invertible.\\
2. All eigenvalues of $\vec{B}$ are necessarily real.\\
3. $\vec{B}-\vec{I}$ is necessarily invertible.\\
4. $\vec{B}$ is necessarily invertible.
\section{solution}
\begin{table}[h!]
\begin{center}
\begin{tabular}{|p{5cm}|p{10cm}|}
\hline
& \\
Given &Let $\vec{A}$ be a real symmetric matrix,\\\\& and $\vec{B}=\vec{I}+i\vec{A}$, where $i^2=-1$.\\
&\\
\hline
&\\
To find  & The correct option.\\
&\\
 \hline
& \\
Proof  & Let us assume,\\\\ & $\lambda$ be the eigen value of $\vec{A}$, as $\vec{A}$ is symmetric matrix.\\
&\\
& $\implies \lambda\epsilon \R$\\
&\\
& Then, $i\lambda$ is an eigen value of $i\vec{A}$\\
&\\
& $\implies 1+i\lambda$ is an eigen value of $\vec{I}+i\vec{A}$ \\
& \\
\hline
\end{tabular}
\end{center}
\end{table}
\newpage
\begin{table}[ht]
\begin{center}
\begin{tabular}{|p{5cm}|p{10cm}|}
\hline
&\\
& Given, $\vec{B}=\vec{I}+i\vec{A}$\\\\
& Therefore, $1+i\lambda$ is an eigen value of $\vec{B}$.\\
&\\
& Hence, 0 can not be the eigen value of $\vec{B}$\\
&\\
\textbf{Property}: $\det\vec{B}$ is equals to 
& $\implies \det\vec{B} \neq 0$\\product of eigen values of $\vec{B}$
& Therefore, $\vec{B}$ is necessarily invertible.
&\\
\hline
&\\
Correct option  & The correct option is $\vec{4}$.\\
&\\
\hline
\end{tabular}
\end{center}
\end{table}
\end{document}

&Given, $\vec{A}$ is a symmetric matrix,\\Let us assume $\lambda$ be the &eigen value of $\vec{A}$\\\\
&\implies $\lambda\epsilon \R$ \\\\ 
&\implies $i\lambda$ is eigen value of $i\vec{A}$\\\\
&\implies $1+i\lambda$ is eigen value of $\vec{I}+i\vec{A}$\\\\
& Given $\vec{B}=\vec{I}+i\vec{A}$\\\\
&Therefore, $1+i\lambda$ is eigen value of $\vec{B}$\\\\
&We know that, $\det\vec{B}$ is equals to product of eigen values of $\vec{B}$\\\\
& Hence 0, can not be the eigen value of $\vec{B}$\\\\
&\implies $\det\vec{B} \neq 0$\\\\
& Therefore, $\vec{B}$ is necessarily invertible.\\\\
&Hence $\vec{4}$. is the correct answer.