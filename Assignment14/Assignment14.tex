\documentclass[journal,12pt,onecolumn]{IEEEtran}

\usepackage{setspace}
\usepackage{gensymb}
\singlespacing


\usepackage[cmex10]{amsmath}

\usepackage{amsthm}

\usepackage{mathrsfs}
\usepackage{txfonts}
\usepackage{stfloats}
\usepackage{bm}
\usepackage{cite}
\usepackage{cases}
\usepackage{subfig}

\usepackage{longtable}
\usepackage{multirow}

\usepackage{enumitem}
\usepackage{mathtools}
\usepackage{steinmetz}
\usepackage{tikz}
\usepackage{circuitikz}
\usepackage{verbatim}
\usepackage{tfrupee}
\usepackage[breaklinks=true]{hyperref}
\usepackage{graphicx}
\usepackage{tkz-euclide}

\usetikzlibrary{calc,math}
\usepackage{listings}
    \usepackage{color}                                            %%
    \usepackage{array}                                            %%
    \usepackage{longtable}                                        %%
    \usepackage{calc}                                             %%
    \usepackage{multirow}                                         %%
    \usepackage{hhline}                                           %%
    \usepackage{ifthen}                                           %%
    \usepackage{lscape}     
\usepackage{multicol}
\usepackage{chngcntr}

\DeclareMathOperator*{\Res}{Res}

\renewcommand\thesection{\arabic{section}}
\renewcommand\thesubsection{\thesection.\arabic{subsection}}
\renewcommand\thesubsubsection{\thesubsection.\arabic{subsubsection}}

\renewcommand\thesectiondis{\arabic{section}}
\renewcommand\thesubsectiondis{\thesectiondis.\arabic{subsection}}
\renewcommand\thesubsubsectiondis{\thesubsectiondis.\arabic{subsubsection}}


\hyphenation{op-tical net-works semi-conduc-tor}
\def\inputGnumericTable{}                                 %%

\lstset{
%language=C,
frame=single, 
breaklines=true,
columns=fullflexible
}
\begin{document}


\newtheorem{theorem}{Theorem}[section]
\newtheorem{problem}{Problem}
\newtheorem{proposition}{Proposition}[section]
\newtheorem{lemma}{Lemma}[section]
\newtheorem{corollary}[theorem]{Corollary}
\newtheorem{example}{Example}[section]
\newtheorem{definition}[problem]{Definition}

\newcommand{\BEQA}{\begin{eqnarray}}
\newcommand{\EEQA}{\end{eqnarray}}
\newcommand{\define}{\stackrel{\triangle}{=}}
\bibliographystyle{IEEEtran}
\providecommand{\mbf}{\mathbf}
\providecommand{\pr}[1]{\ensuremath{\Pr\left(#1\right)}}
\providecommand{\qfunc}[1]{\ensuremath{Q\left(#1\right)}}
\providecommand{\sbrak}[1]{\ensuremath{{}\left[#1\right]}}
\providecommand{\lsbrak}[1]{\ensuremath{{}\left[#1\right.}}
\providecommand{\rsbrak}[1]{\ensuremath{{}\left.#1\right]}}
\providecommand{\brak}[1]{\ensuremath{\left(#1\right)}}
\providecommand{\lbrak}[1]{\ensuremath{\left(#1\right.}}
\providecommand{\rbrak}[1]{\ensuremath{\left.#1\right)}}
\providecommand{\cbrak}[1]{\ensuremath{\left\{#1\right\}}}
\providecommand{\lcbrak}[1]{\ensuremath{\left\{#1\right.}}
\providecommand{\rcbrak}[1]{\ensuremath{\left.#1\right\}}}
\theoremstyle{remark}
\newtheorem{rem}{Remark}
\newcommand{\sgn}{\mathop{\mathrm{sgn}}}
\providecommand{\abs}[1]{\ensuremath{\left\vert#1\right\vert}}
\providecommand{\res}[1]{\Res\displaylimits_{#1}} 
\providecommand{\norm}[1]{\ensuremath{\left\lVert#1\right\rVert}}
%\providecommand{\norm}[1]{\lVert#1\rVert}
\providecommand{\mtx}[1]{\mathbf{#1}}
\providecommand{\mean}[1]{E\ensuremath{\left[ #1 \right]}}
\providecommand{\fourier}{\overset{\mathcal{F}}{ \rightleftharpoons}}
%\providecommand{\hilbert}{\overset{\mathcal{H}}{ \rightleftharpoons}}
\providecommand{\system}{\overset{\mathcal{H}}{ \longleftrightarrow}}
	%\newcommand{\solution}[2]{\textbf{Solution:}{#1}}
\newcommand{\solution}{\noindent \textbf{Solution: }}
\newcommand{\cosec}{\,\text{cosec}\,}
\newcommand{\R}{\mathbb{R}}
\providecommand{\dec}[2]{\ensuremath{\overset{#1}{\underset{#2}{\gtrless}}}}
\newcommand{\myvec}[1]{\ensuremath{\begin{pmatrix}#1\end{pmatrix}}}
\newcommand{\mydet}[1]{\ensuremath{\begin{vmatrix}#1\end{vmatrix}}}
\numberwithin{equation}{subsection}
\makeatletter
\@addtoreset{figure}{problem}
\makeatother
\let\StandardTheFigure\thefigure
\let\vec\mathbf
\renewcommand{\thefigure}{\theproblem}
\def\putbox#1#2#3{\makebox[0in][l]{\makebox[#1][l]{}\raisebox{\baselineskip}[0in][0in]{\raisebox{#2}[0in][0in]{#3}}}}
     \def\rightbox#1{\makebox[0in][r]{#1}}
     \def\centbox#1{\makebox[0in]{#1}}
     \def\topbox#1{\raisebox{-\baselineskip}[0in][0in]{#1}}
     \def\midbox#1{\raisebox{-0.5\baselineskip}[0in][0in]{#1}}
\vspace{3cm}
\onecolumn
\title{EE5609: Matrix Theory\\
          Assignment-14\\}
\author{Major Saurabh Joshi\\MTech Artificial Intelligence\\AI20MTECH13002 }
\maketitle
\bigskip
\renewcommand{\thefigure}{\theenumi}
\renewcommand{\thetable}{\theenumi}
Download codes from 
%
\begin{lstlisting}
https://github.com/saurabh13002/EE5609/tree/master/Assignment14
\end{lstlisting}
%
 
\section{Question}
Let $\vec{M_n(K)}$,denote the space of all n × n matrices with entries in a field $\mathbb{K}$. Fix a non singular matrix $\vec{A}=\vec{(A_{ij})}\epsilon \vec{M_n(K)}$ , and consider the linear map $\vec{T}:\vec{M_n(K)}\rightarrow\vec{M_n(K)}$given by $\vec{T(X)=AX}$.Then\\

\begin{enumerate}
\item trace$(\vec{T})=n\sum_{i=1}^n \vec{A}_{ii} .$
\item trace$(\vec{T})=\sum_{i=1}^n\sum_{j=1}^n \vec{A}_{ij} .$
\item $rank(\vec{T})=$n$^2$.
\item $\vec{T}$ is non singular.
\end{enumerate}
%
\section{Solution}
\begin{longtable}{|p{5cm}|p{13cm}|}
\hline
\textbf{Statement} &\textbf{Solution}\\
\hline
Definition & $\vec{T}:\mathbb{F}^{nxm}\rightarrow \mathbb {F}^{nxm}$ defined as:
\begin{displaymath}\vec{T(X)=AX} \end{displaymath}


where $\vec{A}$ n x n is fixed and is a linear transformation.  \\
\hline
Properties&If $\vec{A}$ is the matrix representation of a linear transformation $\vec{T}$, then\\&
\parbox{12cm}{\begin{align}
nullity(\vec{T})=\text{m}. nullity(\vec{A}) \label{eq1}\\ 
rank(\vec{T})=\text{m}.rank(\vec{A}) \label{eq2}\\
tr(\vec{T})=\text{m}.tr(\vec{A}) \label{eq3}\\
\text{Also,rank of a non singular n x n matrix}, \vec{A}= \text{n} \label{eq4}\\
tr({\vec{A}}) = \sum_{i=1}^n \vec{A}_{ii} \label{eq5}
\end{align}}\\
\hline
Checking $tr$$(\vec{T}).$ &
\parbox{12cm}{\begin{align}
 \text{ from \eqref{eq3}, } tr(\vec{T})=\text{m}.tr(\vec{A}), \\ \text{Since, $\vec{A}$ is a square matrix $\therefore$ m=n }\label{eq6}\\
            \text{also, from \eqref{eq5}, }\\
 \implies tr(\vec{T})   = \text{n}\sum_{i=1}^n \vec{A}_{ii} \label{eq7}\\
           \text{Hence it is a correct option.} \end{align}}\\
\hline
Checking tr$(\vec{T})=\sum_{i=1}^n\sum_{j=1}^n \vec{A}_{ij}.$&
\parbox{12cm}{\begin{align}\text{from \eqref{eq7},}\
    tr(\vec{T})=\text{n}\sum_{i=1}^n \vec{A}_{ii},
    \text{Hence, discarding the option.}\label{eq11}
\end{align}}\\
\hline
 Checking  $rank(\vec{T})=$n$^2$.&
\parbox{12cm}{\begin{align}\text{from \eqref{eq2} and \eqref{eq4},}
\  rank(\vec{T})=\text{m}.rank(\vec{A})\\\text{ from \eqref{eq6},}\
\  rank(\vec{T})=\text{n}.rank(\vec{A})\\
\implies \  rank(\vec{T})=\text{n.n}=\text{n}^2 \label{eq8}\\
\therefore \text{Option 3 is also correct.}
\end{align}}\\
\hline
Checking $\vec{T}$ is non singular.&
\parbox{12cm}{\begin{align}\text{from the given data $\vec{T(X)=AX}$ is a linear map and $\vec{A}$ is non singular .} \\ \text{Hence}, \vec{T} \text{ is non singular.}\label{eq9}
\end{align}}\\
\hline
  Conclusion&
Hence, from \eqref{eq7},\eqref{eq11},\eqref{eq8} and \eqref{eq9} option 1,3 and 4 are the correct answers.\\
\hline
\caption*{Table1:Solution}
\end{longtable}
\end{document}
\text{The rank of a matrix is the maximum number of independent rows (or, the maximum number of independent columns). A square matrix \vec{A},n x n is non-singular only if its rank is equal to n.}\\