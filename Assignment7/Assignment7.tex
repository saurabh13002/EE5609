\documentclass[journal,12pt,twocolumn]{IEEEtran}

\usepackage{setspace}
\usepackage{gensymb}
\singlespacing
\usepackage[cmex10]{amsmath}

\usepackage{amsthm}

\usepackage{mathrsfs}
\usepackage{txfonts}
\usepackage{stfloats}
\usepackage{bm}
\usepackage{cite}
\usepackage{cases}
\usepackage{subfig}

\usepackage{longtable}
\usepackage{multirow}

\usepackage{enumitem}
\usepackage{mathtools}
\usepackage{steinmetz}
\usepackage{tikz}
\usepackage{circuitikz}
\usepackage{verbatim}
\usepackage{tfrupee}
\usepackage[breaklinks=true]{hyperref}
\usepackage{graphicx}
\usepackage{tkz-euclide}

\usetikzlibrary{calc,math}
\usepackage{listings}
    \usepackage{color}                                            %%
    \usepackage{array}                                            %%
    \usepackage{longtable}                                        %%
    \usepackage{calc}                                             %%
    \usepackage{multirow}                                         %%
    \usepackage{hhline}                                           %%
    \usepackage{ifthen}                                           %%
    \usepackage{lscape}     
\usepackage{multicol}
\usepackage{chngcntr}

\DeclareMathOperator*{\Res}{Res}

\renewcommand\thesection{\arabic{section}}
\renewcommand\thesubsection{\thesection.\arabic{subsection}}
\renewcommand\thesubsubsection{\thesubsection.\arabic{subsubsection}}

\renewcommand\thesectiondis{\arabic{section}}
\renewcommand\thesubsectiondis{\thesectiondis.\arabic{subsection}}
\renewcommand\thesubsubsectiondis{\thesubsectiondis.\arabic{subsubsection}}


\hyphenation{op-tical net-works semi-conduc-tor}
\def\inputGnumericTable{}                                 %%

\lstset{
%language=C,
frame=single, 
breaklines=true,
columns=fullflexible
}
\begin{document}


\newtheorem{theorem}{Theorem}[section]
\newtheorem{problem}{Problem}
\newtheorem{proposition}{Proposition}[section]
\newtheorem{lemma}{Lemma}[section]
\newtheorem{corollary}[theorem]{Corollary}
\newtheorem{example}{Example}[section]
\newtheorem{definition}[problem]{Definition}

\newcommand{\BEQA}{\begin{eqnarray}}
\newcommand{\EEQA}{\end{eqnarray}}
\newcommand{\define}{\stackrel{\triangle}{=}}
\bibliographystyle{IEEEtran}
\raggedbottom
\setlength{\parindent}{0pt}
\providecommand{\mbf}{\mathbf}
\providecommand{\pr}[1]{\ensuremath{\Pr\left(#1\right)}}
\providecommand{\qfunc}[1]{\ensuremath{Q\left(#1\right)}}
\providecommand{\sbrak}[1]{\ensuremath{{}\left[#1\right]}}
\providecommand{\lsbrak}[1]{\ensuremath{{}\left[#1\right.}}
\providecommand{\rsbrak}[1]{\ensuremath{{}\left.#1\right]}}
\providecommand{\brak}[1]{\ensuremath{\left(#1\right)}}
\providecommand{\lbrak}[1]{\ensuremath{\left(#1\right.}}
\providecommand{\rbrak}[1]{\ensuremath{\left.#1\right)}}
\providecommand{\cbrak}[1]{\ensuremath{\left\{#1\right\}}}
\providecommand{\lcbrak}[1]{\ensuremath{\left\{#1\right.}}
\providecommand{\rcbrak}[1]{\ensuremath{\left.#1\right\}}}
\theoremstyle{remark}
\newtheorem{rem}{Remark}
\newcommand{\sgn}{\mathop{\mathrm{sgn}}}
\providecommand{\abs}[1]{\left\vert#1\right\vert}
\providecommand{\res}[1]{\Res\displaylimits_{#1}} 
\providecommand{\norm}[1]{\left\lVert#1\right\rVert}
%\providecommand{\norm}[1]{\lVert#1\rVert}
\providecommand{\mtx}[1]{\mathbf{#1}}
\providecommand{\mean}[1]{E\left[ #1 \right]}
\providecommand{\fourier}{\overset{\mathcal{F}}{ \rightleftharpoons}}
%\providecommand{\hilbert}{\overset{\mathcal{H}}{ \rightleftharpoons}}
\providecommand{\system}{\overset{\mathcal{H}}{ \longleftrightarrow}}
	%\newcommand{\solution}[2]{\textbf{Solution:}{#1}}
\newcommand{\solution}{\noindent \textbf{Solution: }}
\newcommand{\cosec}{\,\text{cosec}\,}
\providecommand{\dec}[2]{\ensuremath{\overset{#1}{\underset{#2}{\gtrless}}}}
\newcommand{\myvec}[1]{\ensuremath{\begin{pmatrix}#1\end{pmatrix}}}
\newcommand{\mydet}[1]{\ensuremath{\begin{vmatrix}#1\end{vmatrix}}}
\numberwithin{equation}{subsection}

\makeatletter
\@addtoreset{figure}{problem}
\makeatother
\let\StandardTheFigure\thefigure
\let\vec\mathbf

\renewcommand{\thefigure}{\theproblem}

\def\putbox#1#2#3{\makebox[0in][l]{\makebox[#1][l]{}\raisebox{\baselineskip}[0in][0in]{\raisebox{#2}[0in][0in]{#3}}}}
     \def\rightbox#1{\makebox[0in][r]{#1}}
     \def\centbox#1{\makebox[0in]{#1}}
     \def\topbox#1{\raisebox{-\baselineskip}[0in][0in]{#1}}
     \def\midbox#1{\raisebox{-0.5\baselineskip}[0in][0in]{#1}}
\vspace{3cm}
\title{EE5609: Matrix Theory\\
          Assignment-7\\}
\author{Major Saurabh Joshi\\MTech Artificial Intelligence\\AI20MTECH13002 }
\maketitle
\newpage
%\tableofcontents
\bigskip
\renewcommand{\thefigure}{\theenumi}
\renewcommand{\thetable}{\theenumi}
\begin{abstract}
This  document use elementary row operations to find invertiblile matrix,and hence the inverse. 
\end{abstract}
Download the latex code from 
%
%
%
\begin{lstlisting}
https://github.com/saurabh13002/EE5609/tree/master/Assignment7
\end{lstlisting}
%
\section{Problem}
For each of the two matrices use elementary row operations to discover whether it is invertible, and to find the inverse in case it is invertible.\\

$\vec{A}=\myvec{2&5&-1\\4&-1&2\\6&4&1},
\vec{B}=\myvec{1&-1&2\\3&2&4\\0&1&-2}$

\section{\textbf{Solution}}Given 
\begin{align}
\vec{A}=\myvec{2&5&-1\\4&-1&2\\6&4&1},
\vec{B}=\myvec{1&-1&2\\3&2&4\\0&1&-2}
\end{align}
 By applying row reductions on $\vec{A}$
 \begin{align}
&\myvec{2&5&-1\\4&-1&2\\6&4&1}\xleftrightarrow{R_2=R_2-2R_1}\vec{A}=\myvec{2&5&-1\\0&-11&4\\6&4&1}\\
&\xleftrightarrow{R_3=R_3-3R_1}\myvec{2&5&-1\\0&-11&4\\0&-11&4}\\
&\xleftrightarrow{R_1=\frac{R_1}{2}}\myvec{1&\frac{5}{2}&\frac{-1}{2}\\0&-11&4\\0&-11&4}
\end{align}
\begin{align}
&\xleftrightarrow{R_3=R_3-R_2}\myvec{1&\frac{5}{2}&\frac{-1}{2}\\0&-11&4\\0&0&0}\\
&\xleftrightarrow{R_2=\frac{-R_2}{11}}\myvec{1&\frac{5}{2}&\frac{-1}{2}\\0&1&\frac{-4}{11}\\0&0&0}\\
&\xleftrightarrow{R_1=R_1-\frac{5}{2}R_2}\myvec{1&0&\frac{9}{22}\\0&1&\frac{-4}{11}\\0&0&0}
\end{align}
For a matrix to be invertible, it has to be a matrix of full rank. However the matrix $\vec{A}$ is not of full rank ($Rank(\vec{A})<3$).Therefore $\vec{A}$ is not invertible.

Let us now consider augmented matrix $\vec{B|I}$, By applying row reductions on $\vec{B|I}$
\begin{multline}
&\myvec{1&-1&2&\vrule&1&0&0\\3&2&4&\vrule&0&1&0\\0&1&-2&\vrule&0&0&1}\xleftrightarrow{R_2=R_2-3R_1}\\\myvec{1&-1&2&\vrule&1&0&0\\0&5&-2&\vrule&-3&1&0\\0&1&-2&\vrule&0&0&1}
\end{multline}
\begin{align}
&\xleftrightarrow{R_2=\frac{R_2}{5}}\myvec{1&-1&2&\vrule&1&0&0\\0&1&\frac{-2}{5}&\vrule&\frac{-3}{5}&\frac{1}{5}&0\\0&1&-2&\vrule&0&0&1}\\
&\xleftrightarrow{R_1=R_1+R_2}\myvec{1&0&\frac{8}{5}&\vrule&\frac{2}{5}&\frac{1}{5}&0\\0&1&\frac{-2}{5}&\vrule&\frac{-3}{5}&\frac{1}{5}&0\\0&1&-2&\vrule&0&0&1}\\
&\xleftrightarrow{R_3=R_3-R_2}\myvec{1&0&\frac{8}{5}&\vrule&\frac{2}{5}&\frac{1}{5}&0\\0&1&\frac{-2}{5}&\vrule&\frac{-3}{5}&\frac{1}{5}&0\\0&0&\frac{-8}{5}&\vrule&\frac{3}{5}&\frac{-1}{5}&1}\\
&\xleftrightarrow{R_1=R_1+R_3}\myvec{1&0&0&\vrule&1&0&1\\0&1&\frac{-2}{5}&\vrule&\frac{-3}{5}&\frac{1}{5}&0\\0&0&\frac{-8}{5}&\vrule&\frac{3}{5}&\frac{-1}{5}&1}
\end{align}
\begin{align}
&\xleftrightarrow{R_3=\frac{-5}{8}R_3}\myvec{1&0&0&\vrule&1&0&1\\0&1&\frac{-2}{5}&\vrule&\frac{-3}{5}&\frac{1}{5}&0\\0&0&1&\vrule&\frac{-3}{8}&\frac{1}{8}&\frac{-5}{8}}\\
&\xleftrightarrow{R_3=R_2+\frac{2}{5}R_3}\myvec{1&0&0&\vrule&1&0&1\\0&1&0&\vrule&\frac{-3}{4}&\frac{1}{4}&\frac{-1}{4}\\0&0&1&\vrule&\frac{-3}{8}&\frac{1}{8}&\frac{-5}{8}}\label{eq1}
\end{align}
For a matrix to be invertible, it has to be a matrix of full rank. Here, the matrix $\vec{B}$ is of full rank ($Rank(\vec{B})=3$). Therefore $\vec{B}$ is invertible and the inverse matrix $\vec{B}^{-1}$ can be written from \eqref{eq1}:
\begin{align}
\vec{B}^{-1}=\myvec{1&0&1\\\frac{-3}{4}&\frac{1}{4}&\frac{-1}{4}\\\frac{-3}{8}&\frac{1}{8}&\frac{-5}{8}}  
\end{align}
\section{Observation}
\begin{enumerate}
    \item For a matrix to be invertible, it has to be a matrix of full rank.
    \item For a given matrix $\vec{A}$, if the augmented matrix $\vec{A|I}$ on applying elementary row operations transforms into a matrix of the form $\vec{I|B}$. Hence,the matrix $\vec{A}$ is invertible, and the inverse matrix  $\vec{A}^{-1}$ is given by $\vec{B}$.
    \item If the reduced row echelon form matrix for $\vec{A|I}$ is not of the form $\vec{I|B}$, then the matrix $\vec{A}$ is not invertible.
\end{enumerate}
 \end{document}