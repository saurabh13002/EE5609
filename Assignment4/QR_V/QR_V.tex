\documentclass[journal,12pt,twocolumn]{IEEEtran}
%
\usepackage{setspace}
\usepackage{gensymb}
%\doublespacing
\singlespacing

%\usepackage{graphicx}
%\usepackage{amssymb}
%\usepackage{relsize}
\usepackage[cmex10]{amsmath}
%\usepackage{amsthm}
%\interdisplaylinepenalty=2500
%\savesymbol{iint}
%\usepackage{txfonts}
%\restoresymbol{TXF}{iint}
%\usepackage{wasysym}
\usepackage{amsthm}
%\usepackage{iithtlc}
\usepackage{mathrsfs}
\usepackage{txfonts}
\usepackage{stfloats}
\usepackage{bm}
\usepackage{cite}
\usepackage{cases}
\usepackage{subfig}
%\usepackage{xtab}
\usepackage{longtable}
\usepackage{multirow}
%\usepackage{algorithm}
%\usepackage{algpseudocode}
\usepackage{enumitem}
\usepackage{mathtools}
\usepackage{steinmetz}
\usepackage{tikz}
\usepackage{circuitikz}
\usepackage{verbatim}
\usepackage{tfrupee}
\usepackage[breaklinks=true]{hyperref}
%\usepackage{stmaryrd}
\usepackage{tkz-euclide} % loads  TikZ and tkz-base
%\usetkzobj{all}
\usetikzlibrary{calc,math}
\usepackage{listings}
    \usepackage{color}                                            %%
    \usepackage{array}                                            %%
    \usepackage{longtable}                                        %%
    \usepackage{calc}                                             %%
    \usepackage{multirow}                                         %%
    \usepackage{hhline}                                           %%
    \usepackage{ifthen}                                           %%
  %optionally (for landscape tables embedded in another document): %%
    \usepackage{lscape}     
\usepackage{multicol}
\usepackage{chngcntr}
%\usepackage{enumerate}

%\usepackage{wasysym}
%\newcounter{MYtempeqncnt}
\DeclareMathOperator*{\Res}{Res}
%\renewcommand{\baselinestretch}{2}
\renewcommand\thesection{\arabic{section}}
\renewcommand\thesubsection{\thesection.\arabic{subsection}}
\renewcommand\thesubsubsection{\thesubsection.\arabic{subsubsection}}

\renewcommand\thesectiondis{\arabic{section}}
\renewcommand\thesubsectiondis{\thesectiondis.\arabic{subsection}}
\renewcommand\thesubsubsectiondis{\thesubsectiondis.\arabic{subsubsection}}

% correct bad hyphenation here
\hyphenation{op-tical net-works semi-conduc-tor}
\def\inputGnumericTable{}                                 %%

\lstset{
%language=C,
frame=single, 
breaklines=true,
columns=fullflexible
}
%\lstset{
%language=tex,
%frame=single, 
%breaklines=true
%}

\begin{document}
%
\newtheorem{theorem}{Theorem}[section]
\newtheorem{problem}{Problem}
\newtheorem{proposition}{Proposition}[section]
\newtheorem{lemma}{Lemma}[section]
\newtheorem{corollary}[theorem]{Corollary}
\newtheorem{example}{Example}[section]
\newtheorem{definition}[problem]{Definition}
%\newtheorem{thm}{Theorem}[section] 
%\newtheorem{defn}[thm]{Definition}
%\newtheorem{algorithm}{Algorithm}[section]
%\newtheorem{cor}{Corollary}
\newcommand{\BEQA}{\begin{eqnarray}}
\newcommand{\EEQA}{\end{eqnarray}}
\newcommand{\define}{\stackrel{\triangle}{=}}
\bibliographystyle{IEEEtran}
%\bibliographystyle{ieeetr}
\providecommand{\mbf}{\mathbf}
\providecommand{\pr}[1]{\ensuremath{\Pr\left(#1\right)}}
\providecommand{\qfunc}[1]{\ensuremath{Q\left(#1\right)}}
\providecommand{\sbrak}[1]{\ensuremath{{}\left[#1\right]}}
\providecommand{\lsbrak}[1]{\ensuremath{{}\left[#1\right.}}
\providecommand{\rsbrak}[1]{\ensuremath{{}\left.#1\right]}}
\providecommand{\brak}[1]{\ensuremath{\left(#1\right)}}
\providecommand{\lbrak}[1]{\ensuremath{\left(#1\right.}}
\providecommand{\rbrak}[1]{\ensuremath{\left.#1\right)}}
\providecommand{\cbrak}[1]{\ensuremath{\left\{#1\right\}}}
\providecommand{\lcbrak}[1]{\ensuremath{\left\{#1\right.}}
\providecommand{\rcbrak}[1]{\ensuremath{\left.#1\right\}}}
\theoremstyle{remark}
\newtheorem{rem}{Remark}
\newcommand{\sgn}{\mathop{\mathrm{sgn}}}
\providecommand{\abs}[1]{\ensuremath{\left\vert#1\right\vert}}
\providecommand{\res}[1]{\Res\displaylimits_{#1}} 
\providecommand{\norm}[1]{\ensuremath{\left\lVert#1\right\rVert}}
%\providecommand{\norm}[1]{\lVert#1\rVert}
\providecommand{\mtx}[1]{\mathbf{#1}}
\providecommand{\mean}[1]{\ensuremath{E\left[ #1 \right]}}
\providecommand{\fourier}{\overset{\mathcal{F}}{ \rightleftharpoons}}
%\providecommand{\hilbert}{\overset{\mathcal{H}}{ \rightleftharpoons}}
\providecommand{\system}{\overset{\mathcal{H}}{ \longleftrightarrow}}
	%\newcommand{\solution}[2]{\textbf{Solution:}{#1}}
\newcommand{\solution}{\noindent \textbf{Solution: }}
\newcommand{\cosec}{\,\text{cosec}\,}
\providecommand{\dec}[2]{\ensuremath{\overset{#1}{\underset{#2}{\gtrless}}}}
\newcommand{\myvec}[1]{\ensuremath{\begin{pmatrix}#1\end{pmatrix}}}
\newcommand{\mydet}[1]{\ensuremath{\begin{vmatrix}#1\end{vmatrix}}}
\newcommand\inv[1]{#1\raisebox{1.15ex}{$\scriptscriptstyle-\!1$}}
%\numberwithin{equation}{section}
\numberwithin{equation}{subsection}
%\numberwithin{problem}{section}
%\numberwithin{definition}{section}
\makeatletter
\@addtoreset{figure}{problem}
\makeatother
\let\StandardTheFigure\thefigure
\let\vec\mathbf
%\renewcommand{\thefigure}{\theproblem.\arabic{figure}}
\renewcommand{\thefigure}{\theproblem}
%\setlist[enumerate,1]{before=\renewcommand\theequation{\theenumi.\arabic{equation}}
%\counterwithin{equation}{enumi}
%\renewcommand{\theequation}{\arabic{subsection}.\arabic{equation}}
\def\putbox#1#2#3{\makebox[0in][l]{\makebox[#1][l]{}\raisebox{\baselineskip}[0in][0in]{\raisebox{#2}[0in][0in]{#3}}}}
     \def\rightbox#1{\makebox[0in][r]{#1}}
     \def\centbox#1{\makebox[0in]{#1}}
     \def\topbox#1{\raisebox{-\baselineskip}[0in][0in]{#1}}
     \def\midbox#1{\raisebox{-0.5\baselineskip}[0in][0in]{#1}}
\vspace{3cm}
\title{EE5609: Matrix Theory\\
          Assignment-4\\}
\author{Major Saurabh Joshi\\MTech Artificial Intelligence\\AI20MTECH13002 }
\maketitle
\newpage
%\tableofcontents
\bigskip
\renewcommand{\thefigure}{\theenumi}
\renewcommand{\thetable}{\theenumi}
\begin{abstract}
This  document explains how to factorize a matrix using QR decomposition. 
\end{abstract}
Download the latex-tikz codes from 
%
%
%
\begin{lstlisting}
https://github.com/saurabh13002/EE5609/tree/master/Assignment4/QR_V
\end{lstlisting}
%
\section{Problem}
Perform QR decomposition of \myvec{3 &-\sqrt3\\-\sqrt3 & 1} 
\section{Explanation}
Let $\vec{a}$ and $\vec{b}$ be columns of a $\vec{A}$. Then, the matrix $\vec{A}$ can be decomposed in the form as:
\begin{align}
 \vec{A} = \vec{QR} 
\end{align}
such that  
\begin{align}
	& \vec{Q}\vec{Q}^T= \vec{Q}^T\vec{Q} = \vec{I} \\ 
	& \vec{Q} = \myvec{\vec{u_1} & \vec{u_2}}  \\
	& \vec{R} = \myvec{k_1 & r_1 \\ 0 & k_2} 
\end{align}
where
\begin{align}
& k_1 = \norm{\vec{a}} \label{1} \\
& \vec{u_1} = \frac{\vec{a}}{k_1} \label{2} \\
& r_1 = \frac{\vec{u_1}^T\vec{b}}{\norm{\vec{u_1}}^2} \label{3} \\
& \vec{u_2} = \frac{\vec{b}-r_1\vec{u_1}}{\norm{\vec{b}-r_1\vec{u_1}}} \label{4} \\
& k_2 = \vec{u_2}^T\vec{b} \label{5}
\end{align}
Then, the matrix can be represented as 
\begin{align}
	\myvec{\vec{a} & \vec{b} } = \myvec{\vec{u_1} & \vec{u_2}}\myvec{k_1 & r_1 \\ 0 & k_2} \label{6}
\end{align}
\section{Solution}
Let $\vec{A}$ be the given matrix. Then $\vec{A} = \myvec{3 & -\sqrt{3} \\ -\sqrt{3} & 1}$ and columns of $\vec{A}$ are $\vec{a}$ and $\vec{b}$, where
\begin{align}
	\vec{a} = \myvec{ 3 \\ -\sqrt{3}} \\
	\vec{b} = \myvec{ -\sqrt{3} \\1}
\end{align}
Now, for given matrix from \eqref{1} and \eqref{2}, we have
\begin{align}
&	k_1 = \norm{\vec{a}} = \sqrt{12} =2\sqrt{3}\\
&	\vec{u_1} = \frac{1}{2\sqrt{3}}\myvec{3 \\ -\sqrt{3}}
\end{align}
By, \eqref{3}, we find
\begin{align}
	r_1 = \frac{\frac{1}{2\sqrt{3}}\myvec{3 & -\sqrt{3}}\myvec{-\sqrt{3} \\ 1}}{1} =-2
\end{align}
Now, by \eqref{4}
\begin{align}
	\vec{u_2} = \frac{\myvec{-\sqrt{3} \\ 1}- \frac{-2}{2\sqrt{3}}\myvec{3 \\-\sqrt{3}}}{\norm{\myvec{-\sqrt{3} \\ 1}- \frac{-2}{2\sqrt{3}}\myvec{3 \\-\sqrt{3}}}} = \myvec{0 \\ 0}
\end{align}
Since the vector is zero, it is linearly dependent and we just skip it.
From \eqref{5},
\begin{align}
	k_2 = \myvec{0 & 0}\myvec{-\sqrt{3}\\ 1} = 0
\end{align}
Now,
\begin{align}
	\vec{Q} = \myvec{\frac{\sqrt{3}}{2}&0 \\-\frac{1}{{2}}&0} 
\end{align}

Now, by \eqref{6} we can wrie matrix $\vec{A}$ as
\begin{align}
	{\myvec{3 &-\sqrt3\\-\sqrt3 & 1}}= \myvec{\frac{\sqrt{3}}{2}&0\\-\frac{1}{{2}}&0}
	\myvec{2\sqrt{3} & -2\\0&0}
\end{align}
which is the required $\vec{QR}$ decomposition of $\vec{A}$.







\end{document}