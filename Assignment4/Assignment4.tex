\documentclass[journal,12pt,twocolumn]{IEEEtran}
%
\usepackage{setspace}
\usepackage{gensymb}
%\doublespacing
\singlespacing

%\usepackage{graphicx}
%\usepackage{amssymb}
%\usepackage{relsize}
\usepackage[cmex10]{amsmath}
%\usepackage{amsthm}
%\interdisplaylinepenalty=2500
%\savesymbol{iint}
%\usepackage{txfonts}
%\restoresymbol{TXF}{iint}
%\usepackage{wasysym}
\usepackage{amsthm}
%\usepackage{iithtlc}
\usepackage{mathrsfs}
\usepackage{txfonts}
\usepackage{stfloats}
\usepackage{bm}
\usepackage{cite}
\usepackage{cases}
\usepackage{subfig}
%\usepackage{xtab}
\usepackage{longtable}
\usepackage{multirow}
%\usepackage{algorithm}
%\usepackage{algpseudocode}
\usepackage{enumitem}
\usepackage{mathtools}
\usepackage{steinmetz}
\usepackage{tikz}
\usepackage{circuitikz}
\usepackage{verbatim}
\usepackage{tfrupee}
\usepackage[breaklinks=true]{hyperref}
%\usepackage{stmaryrd}
\usepackage{tkz-euclide} % loads  TikZ and tkz-base
%\usetkzobj{all}
\usetikzlibrary{calc,math}
\usepackage{listings}
    \usepackage{color}                                            %%
    \usepackage{array}                                            %%
    \usepackage{longtable}                                        %%
    \usepackage{calc}                                             %%
    \usepackage{multirow}                                         %%
    \usepackage{hhline}                                           %%
    \usepackage{ifthen}                                           %%
  %optionally (for landscape tables embedded in another document): %%
    \usepackage{lscape}     
\usepackage{multicol}
\usepackage{chngcntr}
%\usepackage{enumerate}

%\usepackage{wasysym}
%\newcounter{MYtempeqncnt}
\DeclareMathOperator*{\Res}{Res}
%\renewcommand{\baselinestretch}{2}
\renewcommand\thesection{\arabic{section}}
\renewcommand\thesubsection{\thesection.\arabic{subsection}}
\renewcommand\thesubsubsection{\thesubsection.\arabic{subsubsection}}

\renewcommand\thesectiondis{\arabic{section}}
\renewcommand\thesubsectiondis{\thesectiondis.\arabic{subsection}}
\renewcommand\thesubsubsectiondis{\thesubsectiondis.\arabic{subsubsection}}

% correct bad hyphenation here
\hyphenation{op-tical net-works semi-conduc-tor}
\def\inputGnumericTable{}                                 %%

\lstset{
%language=C,
frame=single, 
breaklines=true,
columns=fullflexible
}
%\lstset{
%language=tex,
%frame=single, 
%breaklines=true
%}

\begin{document}
%
\newtheorem{theorem}{Theorem}[section]
\newtheorem{problem}{Problem}
\newtheorem{proposition}{Proposition}[section]
\newtheorem{lemma}{Lemma}[section]
\newtheorem{corollary}[theorem]{Corollary}
\newtheorem{example}{Example}[section]
\newtheorem{definition}[problem]{Definition}
%\newtheorem{thm}{Theorem}[section] 
%\newtheorem{defn}[thm]{Definition}
%\newtheorem{algorithm}{Algorithm}[section]
%\newtheorem{cor}{Corollary}
\newcommand{\BEQA}{\begin{eqnarray}}
\newcommand{\EEQA}{\end{eqnarray}}
\newcommand{\define}{\stackrel{\triangle}{=}}
\bibliographystyle{IEEEtran}
%\bibliographystyle{ieeetr}
\providecommand{\mbf}{\mathbf}
\providecommand{\pr}[1]{\ensuremath{\Pr\left(#1\right)}}
\providecommand{\qfunc}[1]{\ensuremath{Q\left(#1\right)}}
\providecommand{\sbrak}[1]{\ensuremath{{}\left[#1\right]}}
\providecommand{\lsbrak}[1]{\ensuremath{{}\left[#1\right.}}
\providecommand{\rsbrak}[1]{\ensuremath{{}\left.#1\right]}}
\providecommand{\brak}[1]{\ensuremath{\left(#1\right)}}
\providecommand{\lbrak}[1]{\ensuremath{\left(#1\right.}}
\providecommand{\rbrak}[1]{\ensuremath{\left.#1\right)}}
\providecommand{\cbrak}[1]{\ensuremath{\left\{#1\right\}}}
\providecommand{\lcbrak}[1]{\ensuremath{\left\{#1\right.}}
\providecommand{\rcbrak}[1]{\ensuremath{\left.#1\right\}}}
\theoremstyle{remark}
\newtheorem{rem}{Remark}
\newcommand{\sgn}{\mathop{\mathrm{sgn}}}
\providecommand{\abs}[1]{\ensuremath{\left\vert#1\right\vert}}
\providecommand{\res}[1]{\Res\displaylimits_{#1}} 
\providecommand{\norm}[1]{\ensuremath{\left\lVert#1\right\rVert}}
%\providecommand{\norm}[1]{\lVert#1\rVert}
\providecommand{\mtx}[1]{\mathbf{#1}}
\providecommand{\mean}[1]{\ensuremath{E\left[ #1 \right]}}
\providecommand{\fourier}{\overset{\mathcal{F}}{ \rightleftharpoons}}
%\providecommand{\hilbert}{\overset{\mathcal{H}}{ \rightleftharpoons}}
\providecommand{\system}{\overset{\mathcal{H}}{ \longleftrightarrow}}
	%\newcommand{\solution}[2]{\textbf{Solution:}{#1}}
\newcommand{\solution}{\noindent \textbf{Solution: }}
\newcommand{\cosec}{\,\text{cosec}\,}
\providecommand{\dec}[2]{\ensuremath{\overset{#1}{\underset{#2}{\gtrless}}}}
\newcommand{\myvec}[1]{\ensuremath{\begin{pmatrix}#1\end{pmatrix}}}
\newcommand{\mydet}[1]{\ensuremath{\begin{vmatrix}#1\end{vmatrix}}}
\newcommand\inv[1]{#1\raisebox{1.15ex}{$\scriptscriptstyle-\!1$}}
%\numberwithin{equation}{section}
\numberwithin{equation}{subsection}
%\numberwithin{problem}{section}
%\numberwithin{definition}{section}
\makeatletter
\@addtoreset{figure}{problem}
\makeatother
\let\StandardTheFigure\thefigure
\let\vec\mathbf
%\renewcommand{\thefigure}{\theproblem.\arabic{figure}}
\renewcommand{\thefigure}{\theproblem}
%\setlist[enumerate,1]{before=\renewcommand\theequation{\theenumi.\arabic{equation}}
%\counterwithin{equation}{enumi}
%\renewcommand{\theequation}{\arabic{subsection}.\arabic{equation}}
\def\putbox#1#2#3{\makebox[0in][l]{\makebox[#1][l]{}\raisebox{\baselineskip}[0in][0in]{\raisebox{#2}[0in][0in]{#3}}}}
     \def\rightbox#1{\makebox[0in][r]{#1}}
     \def\centbox#1{\makebox[0in]{#1}}
     \def\topbox#1{\raisebox{-\baselineskip}[0in][0in]{#1}}
     \def\midbox#1{\raisebox{-0.5\baselineskip}[0in][0in]{#1}}
\vspace{3cm}
\title{EE5609: Matrix Theory\\
          Assignment-4\\}
\author{Major Saurabh Joshi\\MTech Artificial Intelligence\\AI20MTECH13002 }
\maketitle
\newpage
%\tableofcontents
\bigskip
\renewcommand{\thefigure}{\theenumi}
\renewcommand{\thetable}{\theenumi}
\begin{abstract}
This document contains solution to determine the conic representing the given equation. 
\end{abstract}
Download the latex-tikz codes from 
%
%
%
\begin{lstlisting}
https://github.com/saurabh13002/EE5609/tree/master/Assignment4
\end{lstlisting}
%
\section{Problem}
What conic does the following equation represent. 
\begin{align*}
y^2-2\sqrt{3}xy+3x^2+6x-4y+5 = 0
\end{align*}
Find the center and equation refereed to centre.
\section{Solution}
The general second degree equation can be expressed as follows,
\begin{align}
\vec{x^T}\vec{V}\vec{x}+2\vec{u^T}\vec{x}+f=0\label{eqmain}
\end{align}
From the given second degree equation we get,
\begin{align}
\vec{V} &= \myvec{3&-\sqrt{3}\\-\sqrt{3}&1}\\ \label{given1}
\vec{u} &= \myvec{3\\-2}\\ 
f &= 5 \label{given2}
\end{align}
Expanding the determinant of $\vec{V}$ we observe, 
\begin{align}
\mydet{3&-\sqrt{3}\\-\sqrt{3}&1} = 0 \label{eq2.1}
\end{align}
Also
\begin{align}
    \mydet{\vec{V}&\vec{u} \\\vec{u}^T & f}=
    \mydet{3&-\sqrt{3}&3\\-\sqrt{3}&1 &-2\\3 &-2&5}
    \neq 0\label{eq2.2}\end{align}
Hence from \eqref{eq2.1} and \eqref{eq2.2} we conclude that given equation is a parabola. The characteristic equation of $\vec{V}$ is given as follows,
\begin{align}
\mydet{\vec{V}-\lambda\vec{I}} = \mydet{3-\lambda &-\sqrt{3}\\-\sqrt{3}& 1-\lambda} &= 0\\
\implies \lambda^2-4\lambda &= 0\label{eqchar}
\end{align}
Hence the characteristic equation of $\vec{V}$ is given by \eqref{eqchar}. The roots of \eqref{eqchar} i.e the eigenvalues are given by
\begin{align}
\lambda_1=0, \lambda_2=4\label{eqeigenvals}    
\end{align}
The eigen vector $\vec{p}$ is defined as, 
\begin{align}
\vec{V}\vec{p} &= \lambda\vec{p}\\
\implies\brak{\vec{V}-\lambda\vec{I}}\vec{p}&=0 \label{eqev}
\end{align}
for $\lambda_1=0$,
\begin{align}
\brak{\vec{V}-\lambda_1\vec{I}}&=\myvec{3&-\sqrt{3}\\-\sqrt{3}&1}\xleftrightarrow[R_1=\frac{1}{\sqrt{3}}R_1]{R_2=R_1+R_2}\myvec{\sqrt{3}&-1\\0&0}\label{eq2.3.0}
\end{align}
Substiuting equation \ref{eq2.3.0} in equation \ref{eqev} and upon normalizing we get we get
\begin{align}
\implies\vec{p_1}&=\myvec{1/2\\\sqrt{3}/2} \label{eq2.3}
\end{align}
Again, for $\lambda_2=4$,
\begin{align}
\brak{\vec{V}-\lambda_2\vec{I}}&=\myvec{-1&-\sqrt{3}\\-\sqrt{3}&-3}\xleftrightarrow[R_1=-\sqrt{3}R_1]{R_2=-\sqrt{3}R_1+R_2}\myvec{1&\sqrt{3}\\0&0} \label{eq2.3.1}
\end{align}
Substiuting equation \ref{eq2.3.1} in equation  \ref{eqev} and upon normalizing we get
\begin{align}
        \vec{p_2}&=\myvec{-\sqrt{3}/2 \\1/2} \label{eqp1}
\end{align}
The matrix \vec{P},
\begin{align}
\vec{P}&=\myvec{\vec{p_1}&\vec{p_2}}=\myvec{1/2&-\sqrt{3}/2\\\sqrt{3}/2&1/2} \\
\vec{D}&=\myvec{0&0\\0&4}
\end{align}
\begin{align}
    \eta=2\vec{p_1}^T\vec{u}=3-2\sqrt{3} 
\end{align}
The focal length of the parabola is given by:
\begin{align}
    \abs{\frac{\eta}{\lambda_2}} 
    = \abs{\frac{3-2\sqrt{3}}{4}} = 0.116
\end{align}
When $\mydet{\vec{V}}=0$, \eqref{eqmain} can be written as
\begin{align}
    \vec{y^T}\vec{D}\vec{y}&=-\eta\myvec{1&0}\vec{y}\label{eq2.4}
    \intertext{And the vertex $\vec{c}$ is given by }
    \myvec{\vec{u^T}+\frac{\eta}{2}\vec{p_1^T} \\ \vec{V}}\vec{c}=
    \myvec{-f \\ \frac{\eta}{2}\vec{p_1}-\vec{u}}\label{eqa} 
\end{align}
Substituting the found values
\begin{align}
\vec{u}^T + \frac{\eta}{2}\vec{p_1}^T = \myvec{3&-2}+\frac{3-2\sqrt{3}}{2}\myvec{\frac{1}{2}&\frac{\sqrt{3}}{2}}\\
\implies\vec{u}^T + \frac{\eta}{2}\vec{p_1}^T =\myvec{\frac{15-2\sqrt{3}}{4}&\frac{-14+3\sqrt{3}}{4}}\label{eq2.23} \\
\frac{\eta}{2} \vec{p_1} -\vec{u}= \myvec{\frac{-9-2\sqrt{3}}{4}\\ \frac{2+3\sqrt{3}}{4}}\label{eq2.24}
\end{align}
using equations \eqref{given1},\eqref{given2},\eqref{eq2.3},\eqref{eq2.23},\eqref{eq2.24} and \eqref{eq2.3} in \eqref{eqa}
\begin{align}
    \myvec{\frac{15-2\sqrt{3}}{4}&\frac{-14+3\sqrt{3}}{4}\\3 & -\sqrt{3}\\ -\sqrt{3}& 1}\vec{c} =\myvec{ -5\\\frac{-9-2\sqrt{3}}{4}\\ \frac{2+3\sqrt{3}}{4}}\label{eqcen}
\end{align}
By performing row reductions on augmented matrix
\begin{multline}
\myvec{\frac{15-2\sqrt{3}}{4}&\frac{-14+3\sqrt{3}}{4}&-5\\3 & -\sqrt{3}&\frac{(-9-2\sqrt{3})}{4}\\ -\sqrt{3}& 1&\frac{2+3\sqrt{3}}{4}}{R_2\xleftrightarrow[]{}{R_1}}\\
\myvec{3 & -\sqrt{3}&\frac{(-9-2\sqrt{3})}{4}\\\frac{15-2\sqrt{3}}{4}&\frac{-14+3\sqrt{3}}{4}&-5\\-\sqrt{3}& 1&\frac{2+3\sqrt{3}}{4} }
\end{multline}
\begin{multline}
\myvec{3 & -\sqrt{3}&\frac{(-9-2\sqrt{3})}{4}\\\frac{15-2\sqrt{3}}{4}&\frac{-14+3\sqrt{3}}{4}&-5\\-\sqrt{3}& 1&\frac{2+3\sqrt{3}}{4}}
\xleftrightarrow[]{R_2\leftarrow R_2-\frac{15-2\sqrt{3}}{12}R_1}\\
\myvec{3 & -\sqrt{3}&\frac{(-9-2\sqrt{3})}{4}\\0 & 2(\sqrt{3}-2) & \frac{(4\sqrt{3}-39)}{16}\\\sqrt{3}& 1&\frac{2+3\sqrt{3}}{4}}
\end{multline}
Therefore, 
\begin{multline}
 \myvec{3 & -\sqrt{3}&\frac{(-9-2\sqrt{3})}{4}\\0&2(\sqrt{3}-2)&(4\sqrt{3}-39)\\-\sqrt{3}& 1&\frac{(2+3\sqrt{3})}{4}}\xleftrightarrow[]{R_3\leftarrow R_3+\frac{1}{\sqrt{3}}R_1}\\
\myvec{3 & -\sqrt{3}&\frac{(-9-2\sqrt{3})}{4}\\0 & 2(\sqrt{3}-2) & \frac{(4\sqrt{3}-39)}{16}\\0& 0&0}
\end{multline}
On solving for values of $\vec{c}$ from the augmented matrix we get
\begin{multline}
\myvec{3 & -1.732&-3.11\\0 & -0.535 & -2\\0& 0&0}
\end{multline}
Hence $\vec{c}$=$\myvec{1.11\\3.73}$.The vertex of parabola at (1.11, 3.73).
\renewcommand{\thefigure}{1}
\begin{figure}[!ht]
    \centering
    \includegraphics[width=\columnwidth]{parabola.png}
    \caption{Parabola with the center c}
    \label{Fig:1}
\end{figure}
\end{document}
 