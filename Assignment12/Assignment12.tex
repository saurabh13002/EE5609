\documentclass[journal,12pt,onecolumn]{IEEEtran}

\usepackage{setspace}
\usepackage{gensymb}
\singlespacing


\usepackage[cmex10]{amsmath}

\usepackage{amsthm}

\usepackage{mathrsfs}
\usepackage{txfonts}
\usepackage{stfloats}
\usepackage{bm}
\usepackage{cite}
\usepackage{cases}
\usepackage{subfig}

\usepackage{longtable}
\usepackage{multirow}

\usepackage{enumitem}
\usepackage{mathtools}
\usepackage{steinmetz}
\usepackage{tikz}
\usepackage{circuitikz}
\usepackage{verbatim}
\usepackage{tfrupee}
\usepackage[breaklinks=true]{hyperref}
\usepackage{graphicx}
\usepackage{tkz-euclide}

\usetikzlibrary{calc,math}
\usepackage{listings}
    \usepackage{color}                                            %%
    \usepackage{array}                                            %%
    \usepackage{longtable}                                        %%
    \usepackage{calc}                                             %%
    \usepackage{multirow}                                         %%
    \usepackage{hhline}                                           %%
    \usepackage{ifthen}                                           %%
    \usepackage{lscape}     
\usepackage{multicol}
\usepackage{chngcntr}

\DeclareMathOperator*{\Res}{Res}

\renewcommand\thesection{\arabic{section}}
\renewcommand\thesubsection{\thesection.\arabic{subsection}}
\renewcommand\thesubsubsection{\thesubsection.\arabic{subsubsection}}

\renewcommand\thesectiondis{\arabic{section}}
\renewcommand\thesubsectiondis{\thesectiondis.\arabic{subsection}}
\renewcommand\thesubsubsectiondis{\thesubsectiondis.\arabic{subsubsection}}


\hyphenation{op-tical net-works semi-conduc-tor}
\def\inputGnumericTable{}                                 %%

\lstset{
%language=C,
frame=single, 
breaklines=true,
columns=fullflexible
}
\begin{document}


\newtheorem{theorem}{Theorem}[section]
\newtheorem{problem}{Problem}
\newtheorem{proposition}{Proposition}[section]
\newtheorem{lemma}{Lemma}[section]
\newtheorem{corollary}[theorem]{Corollary}
\newtheorem{example}{Example}[section]
\newtheorem{definition}[problem]{Definition}

\newcommand{\BEQA}{\begin{eqnarray}}
\newcommand{\EEQA}{\end{eqnarray}}
\newcommand{\define}{\stackrel{\triangle}{=}}
\bibliographystyle{IEEEtran}
\providecommand{\mbf}{\mathbf}
\providecommand{\pr}[1]{\ensuremath{\Pr\left(#1\right)}}
\providecommand{\qfunc}[1]{\ensuremath{Q\left(#1\right)}}
\providecommand{\sbrak}[1]{\ensuremath{{}\left[#1\right]}}
\providecommand{\lsbrak}[1]{\ensuremath{{}\left[#1\right.}}
\providecommand{\rsbrak}[1]{\ensuremath{{}\left.#1\right]}}
\providecommand{\brak}[1]{\ensuremath{\left(#1\right)}}
\providecommand{\lbrak}[1]{\ensuremath{\left(#1\right.}}
\providecommand{\rbrak}[1]{\ensuremath{\left.#1\right)}}
\providecommand{\cbrak}[1]{\ensuremath{\left\{#1\right\}}}
\providecommand{\lcbrak}[1]{\ensuremath{\left\{#1\right.}}
\providecommand{\rcbrak}[1]{\ensuremath{\left.#1\right\}}}
\theoremstyle{remark}
\newtheorem{rem}{Remark}
\newcommand{\sgn}{\mathop{\mathrm{sgn}}}
\providecommand{\abs}[1]{\ensuremath{\left\vert#1\right\vert}}
\providecommand{\res}[1]{\Res\displaylimits_{#1}} 
\providecommand{\norm}[1]{\ensuremath{\left\lVert#1\right\rVert}}
%\providecommand{\norm}[1]{\lVert#1\rVert}
\providecommand{\mtx}[1]{\mathbf{#1}}
\providecommand{\mean}[1]{E\ensuremath{\left[ #1 \right]}}
\providecommand{\fourier}{\overset{\mathcal{F}}{ \rightleftharpoons}}
%\providecommand{\hilbert}{\overset{\mathcal{H}}{ \rightleftharpoons}}
\providecommand{\system}{\overset{\mathcal{H}}{ \longleftrightarrow}}
	%\newcommand{\solution}[2]{\textbf{Solution:}{#1}}
\newcommand{\solution}{\noindent \textbf{Solution: }}
\newcommand{\cosec}{\,\text{cosec}\,}
\newcommand{\R}{\mathbb{R}}
\providecommand{\dec}[2]{\ensuremath{\overset{#1}{\underset{#2}{\gtrless}}}}
\newcommand{\myvec}[1]{\ensuremath{\begin{pmatrix}#1\end{pmatrix}}}
\newcommand{\mydet}[1]{\ensuremath{\begin{vmatrix}#1\end{vmatrix}}}
\numberwithin{equation}{subsection}
\makeatletter
\@addtoreset{figure}{problem}
\makeatother
\let\StandardTheFigure\thefigure
\let\vec\mathbf
\renewcommand{\thefigure}{\theproblem}
\def\putbox#1#2#3{\makebox[0in][l]{\makebox[#1][l]{}\raisebox{\baselineskip}[0in][0in]{\raisebox{#2}[0in][0in]{#3}}}}
     \def\rightbox#1{\makebox[0in][r]{#1}}
     \def\centbox#1{\makebox[0in]{#1}}
     \def\topbox#1{\raisebox{-\baselineskip}[0in][0in]{#1}}
     \def\midbox#1{\raisebox{-0.5\baselineskip}[0in][0in]{#1}}
\vspace{3cm}
\onecolumn
\title{EE5609: Matrix Theory\\
          Assignment-12\\}
\author{Major Saurabh Joshi\\MTech Artificial Intelligence\\AI20MTECH13002 }
\maketitle
\bigskip
\renewcommand{\thefigure}{\theenumi}
\renewcommand{\thetable}{\theenumi}
Download codes from 
%
\begin{lstlisting}
https://github.com/saurabh13002/EE5609/tree/master/Assignment12
\end{lstlisting}
%
 
\section{Question}
Which of the following subsets of $\mathbb{R}^4$ is a basis of $\mathbb{R}^4$ ?\\
$\vec{B_1}$=$\{1000,1100,1110,1111\}$\\$\vec{B_2}$= $\{1000,1200,1230,1234\}$\\$\vec{B_3}$=$\{1200,0011,2100,-5500\}$
\begin{enumerate}
\item $\vec{B_1}$ and $\vec{B_2}$ but not $\vec{B_3}.$ 
\item $\vec{B_1}$,$\vec{B_2}$, and $\vec{B_3}.$
\item $\vec{B_1}$ and $\vec{B_3}$ but not $\vec{B_2}.$
\item Only $\vec{B_1}.$ 
\end{enumerate}
%
\section{Solution}
\begin{longtable}{|p{5cm}|p{13cm}|}
\hline
\textbf{Statement} &\textbf{Solution}\\
\hline
Definition & Let $\vec{V}$  be a vector space. Then  $\{\vec{v}_{1},\cdots ,\vec{v}_{n}\}$  is called a basis for  $\vec{V}$ if the following conditions hold. \\
& \parbox{12cm}{\begin{align}
    \text{span} \{\vec{v}_{1},\cdots ,\vec{v}_{n}\}=\vec{V}\\
    \{\vec{v}_{1},\cdots ,\vec{v}_{n}\} \text{ is linearly independent}
\end{align}}\\
\hline
Given&
\parbox{12cm}{\begin{align}
\vec{B_1}=\myvec{1&1&1&1\\0&1&1&1\\0&0&1&1\\0&0&0&1},  \vec{B_2}=\myvec{1&1&1&1\\0&2&2&2\\0&0&3&3\\0&0&0&4},
\vec{B_3}=\myvec{1&0&2&-5\\2&0&1&5\\0&1&0&0\\0&1&0&0}
\end{align}}\\
\hline
Checking $\vec{B_1}$  &
\parbox{12cm}{\begin{align}\text{Checking for linear independence.}
 \text{Upon row reducing }\vec{B_1}\\ \myvec{1&1&1&1\\0&1&1&1\\0&0&1&1\\0&0&0&1}\xleftrightarrow[]{R_1\rightarrow R_1-R_2,R_2\rightarrow R_2-R_3,R_3\rightarrow R_3-R_4}
  \myvec{1&0&0&0\\0&1&0&0\\0&0&1&0\\0&0&0&1}
\end{align}}\\
&Clearly Rank of $\vec{B_1}$ is 4,ie full rank.Hence it forms a Basis.\\
\hline
Checking $\vec{B_2}$&
\parbox{12cm}{\begin{align}\text{Checking for linear independence.}
  \text{Upon row reducing }\vec{B_2}\\
    \myvec{1&1&1&1\\0&2&2&2\\0&0&3&3\\0&0&0&4}\xleftrightarrow[]{R_2\rightarrow\frac{R_2}{2}, R_1\rightarrow R_1-R_2,R_3\rightarrow \frac{R_3}{3},R_2\rightarrow R_2-R_3,R_4\rightarrow \frac{R_4}{4},R_3\rightarrow  R_3-R_4}\myvec{1&0&0&0\\0&1&0&0\\0&0&1&0\\0&0&0&1}
\end{align}}\\
& Rank of $\vec{B_2}$ is 4, ie full rank.Hence it also forms a Basis.\\
\hline
 Checking $\vec{B_3}$&
\parbox{12cm}{\begin{align}\text{Checking for linear independence.}
  \text{Upon row reducing }\vec{B_3}\\
    \myvec{1&0&2&-5\\2&0&1&5\\0&1&0&0\\0&1&0&0}\xleftrightarrow[]{R_2\rightarrow R_2-2R_1, R_4\rightarrow R_4-R_2,R_3\rightarrow -\frac{R_3}{3},R_1\rightarrow R_1-2R_3}\myvec{1&0&0&5\\0&1&0&0\\0&0&1&-5\\0&0&0&0}
\end{align}}\\
& Rank of $\vec{B_3}$ is 3, ie not full rank.Hence it does not forms a Basis.\\
\hline
 Conclusion&
Hence option 1, ie  $\vec{B_1}$,$\vec{B_2}$ and not  $\vec{B_3}$ is the correct answer.\\
\hline
\caption*{Table1:Solution}
\end{longtable}
\end{document}
