\documentclass[journal,12pt,twocolumn]{IEEEtran}
%
\usepackage{setspace}
\usepackage{gensymb}
%\doublespacing
\singlespacing

%\usepackage{graphicx}
%\usepackage{amssymb}
%\usepackage{relsize}
\usepackage[cmex10]{amsmath}
%\usepackage{amsthm}
%\interdisplaylinepenalty=2500
%\savesymbol{iint}
%\usepackage{txfonts}
%\restoresymbol{TXF}{iint}
%\usepackage{wasysym}
\usepackage{amsthm}
%\usepackage{iithtlc}
\usepackage{mathrsfs}
\usepackage{txfonts}
\usepackage{stfloats}
\usepackage{bm}
\usepackage{cite}
\usepackage{cases}
\usepackage{subfig}
%\usepackage{xtab}
\usepackage{longtable}
\usepackage{multirow}
%\usepackage{algorithm}
%\usepackage{algpseudocode}
\usepackage{enumitem}
\usepackage{mathtools}
\usepackage{steinmetz}
\usepackage{tikz}
\usepackage{circuitikz}
\usepackage{verbatim}
\usepackage{tfrupee}
\usepackage[breaklinks=true]{hyperref}
%\usepackage{stmaryrd}
\usepackage{tkz-euclide} % loads  TikZ and tkz-base
%\usetkzobj{all}
\usetikzlibrary{calc,math}
\usepackage{listings}
    \usepackage{color}                                            %%
    \usepackage{array}                                            %%
    \usepackage{longtable}                                        %%
    \usepackage{calc}                                             %%
    \usepackage{multirow}                                         %%
    \usepackage{hhline}                                           %%
    \usepackage{ifthen}                                           %%
  %optionally (for landscape tables embedded in another document): %%
    \usepackage{lscape}     
\usepackage{multicol}
\usepackage{chngcntr}
%\usepackage{enumerate}

%\usepackage{wasysym}
%\newcounter{MYtempeqncnt}
\DeclareMathOperator*{\Res}{Res}
%\renewcommand{\baselinestretch}{2}
\renewcommand\thesection{\arabic{section}}
\renewcommand\thesubsection{\thesection.\arabic{subsection}}
\renewcommand\thesubsubsection{\thesubsection.\arabic{subsubsection}}

\renewcommand\thesectiondis{\arabic{section}}
\renewcommand\thesubsectiondis{\thesectiondis.\arabic{subsection}}
\renewcommand\thesubsubsectiondis{\thesubsectiondis.\arabic{subsubsection}}

% correct bad hyphenation here
\hyphenation{op-tical net-works semi-conduc-tor}
\def\inputGnumericTable{}                                 %%

\lstset{
%language=C,
frame=single, 
breaklines=true,
columns=fullflexible
}
%\lstset{
%language=tex,
%frame=single, 
%breaklines=true
%}

\begin{document}
%
\newtheorem{theorem}{Theorem}[section]
\newtheorem{problem}{Problem}
\newtheorem{proposition}{Proposition}[section]
\newtheorem{lemma}{Lemma}[section]
\newtheorem{corollary}[theorem]{Corollary}
\newtheorem{example}{Example}[section]
\newtheorem{definition}[problem]{Definition}
%\newtheorem{thm}{Theorem}[section] 
%\newtheorem{defn}[thm]{Definition}
%\newtheorem{algorithm}{Algorithm}[section]
%\newtheorem{cor}{Corollary}
\newcommand{\BEQA}{\begin{eqnarray}}
\newcommand{\EEQA}{\end{eqnarray}}
\newcommand{\define}{\stackrel{\triangle}{=}}
\bibliographystyle{IEEEtran}
%\bibliographystyle{ieeetr}
\providecommand{\mbf}{\mathbf}
\providecommand{\pr}[1]{\ensuremath{\Pr\left(#1\right)}}
\providecommand{\qfunc}[1]{\ensuremath{Q\left(#1\right)}}
\providecommand{\sbrak}[1]{\ensuremath{{}\left[#1\right]}}
\providecommand{\lsbrak}[1]{\ensuremath{{}\left[#1\right.}}
\providecommand{\rsbrak}[1]{\ensuremath{{}\left.#1\right]}}
\providecommand{\brak}[1]{\ensuremath{\left(#1\right)}}
\providecommand{\lbrak}[1]{\ensuremath{\left(#1\right.}}
\providecommand{\rbrak}[1]{\ensuremath{\left.#1\right)}}
\providecommand{\cbrak}[1]{\ensuremath{\left\{#1\right\}}}
\providecommand{\lcbrak}[1]{\ensuremath{\left\{#1\right.}}
\providecommand{\rcbrak}[1]{\ensuremath{\left.#1\right\}}}
\theoremstyle{remark}
\newtheorem{rem}{Remark}
\newcommand{\sgn}{\mathop{\mathrm{sgn}}}
\providecommand{\abs}[1]{\ensuremath{\left\vert#1\right\vert}}
\providecommand{\res}[1]{\Res\displaylimits_{#1}} 
\providecommand{\norm}[1]{\ensuremath{\left\lVert#1\right\rVert}}
%\providecommand{\norm}[1]{\lVert#1\rVert}
\providecommand{\mtx}[1]{\mathbf{#1}}
\providecommand{\mean}[1]{\ensuremath{E\left[ #1 \right]}}
\providecommand{\fourier}{\overset{\mathcal{F}}{ \rightleftharpoons}}
%\providecommand{\hilbert}{\overset{\mathcal{H}}{ \rightleftharpoons}}
\providecommand{\system}{\overset{\mathcal{H}}{ \longleftrightarrow}}
	%\newcommand{\solution}[2]{\textbf{Solution:}{#1}}
\newcommand{\solution}{\noindent \textbf{Solution: }}
\newcommand{\cosec}{\,\text{cosec}\,}
\providecommand{\dec}[2]{\ensuremath{\overset{#1}{\underset{#2}{\gtrless}}}}
\newcommand{\myvec}[1]{\ensuremath{\begin{pmatrix}#1\end{pmatrix}}}
\newcommand{\mydet}[1]{\ensuremath{\begin{vmatrix}#1\end{vmatrix}}}
\newcommand\inv[1]{#1\raisebox{1.15ex}{$\scriptscriptstyle-\!1$}}
%\numberwithin{equation}{section}
\numberwithin{equation}{subsection}
%\numberwithin{problem}{section}
%\numberwithin{definition}{section}
\makeatletter
\@addtoreset{figure}{problem}
\makeatother
\let\StandardTheFigure\thefigure
\let\vec\mathbf
%\renewcommand{\thefigure}{\theproblem.\arabic{figure}}
\renewcommand{\thefigure}{\theproblem}
%\setlist[enumerate,1]{before=\renewcommand\theequation{\theenumi.\arabic{equation}}
%\counterwithin{equation}{enumi}
%\renewcommand{\theequation}{\arabic{subsection}.\arabic{equation}}
\def\putbox#1#2#3{\makebox[0in][l]{\makebox[#1][l]{}\raisebox{\baselineskip}[0in][0in]{\raisebox{#2}[0in][0in]{#3}}}}
     \def\rightbox#1{\makebox[0in][r]{#1}}
     \def\centbox#1{\makebox[0in]{#1}}
     \def\topbox#1{\raisebox{-\baselineskip}[0in][0in]{#1}}
     \def\midbox#1{\raisebox{-0.5\baselineskip}[0in][0in]{#1}}
\vspace{3cm}
\title{EE5609: Matrix Theory\\
          Assignment-5\\}
\author{Major Saurabh Joshi\\MTech Artificial Intelligence\\AI20MTECH13002 }
\maketitle
\newpage
%\tableofcontents
\bigskip
\renewcommand{\thefigure}{\theenumi}
\renewcommand{\thetable}{\theenumi}
\begin{abstract}
This  document finds the coordinates of foot of perpendicular using Singular Value Decomposition. 
\end{abstract}
Download the latex code from 
%
%
%
\begin{lstlisting}
https://github.com/saurabh13002/EE5609/tree/master/Assignment5_SVD
\end{lstlisting}
%
\section{Problem}
Perform SVD and find the foot of the perpendicular from any point outside the plane.
 \myvec{14&-3&-18}$\vec{x}$ = -1
 \section{\textbf{Solution}}
 Let the point be $\myvec{2\\3\\-5}$,
First we find orthogonal vectors $\vec{m_1}$ and $\vec{m_2}$ to the given normal vector $\vec{n}$. Let, $\vec{m}$ = $\myvec{a\\b\\c}$, then
\begin{align}
\vec{m^T}\vec{n} &= 0\\
\implies\myvec{a&b&c}\myvec{14\\-3\\18} &= 0\\
\implies 14a-3b+18c &= 0
\end{align}
Putting a=1 and b=0 we get,
\begin{align}
\vec{m_1} &= \myvec{1\\0\\\frac{-7}{9}}\end{align}
Putting a=0 and b=1 we get,
\begin{align}
\vec{m_2} &= \myvec{0\\1\\\frac{1}{6}}
\end{align}
Now we solve the equation,
\begin{align}
\vec{M}\vec{x} &= \vec{b}\label{eq1}\\
\intertext{Putting values in \eqref{eq1},}
\myvec{1&0\\0&1\\\frac{-7}{9}&\frac{1}{6}}\vec{x} &= \myvec{2\\3\\-5}\label{eq2}
\end{align}
In order to solve \eqref{eq2},  perform Singular Value Decomposition on $\vec{M}$ as follows,
\begin{align}
\vec{M}=\vec{U}\vec{S}\vec{V}^T\label{eq100}
\end{align}
Where the columns of $\vec{V}$ are the eigen vectors of $\vec{M}^T\vec{M}$ ,the columns of $\vec{U}$ are the eigen vectors of $\vec{M}\vec{M}^T$ and $\vec{S}$ is diagonal matrix of singular value of eigenvalues of $\vec{M}^T\vec{M}$.
\begin{align}
\vec{M}^T\vec{M}=\myvec{\frac{130}{81}&\frac{-7}{54}\\\frac{-7}{54}&\frac{37}{36}}\label{eqMTM}\\
\vec{M}\vec{M}^T=\myvec{1&0&\frac{-7}{9}\\0&1&\frac{1}{6}\\\frac{-7}{9}&\frac{1}{6}&\frac{205}{324}}
\end{align}
From \eqref{eq1} putting \eqref{eq100} we get,
\begin{align}\label{eqX}
\vec{U}\vec{S}\vec{V}^T\vec{x} & = \vec{b}\\
\implies\vec{x} &= \vec{V}\vec{S_+}\vec{U^T}\vec{b}
\end{align}
Where $\vec{S_+}$ is Moore-Penrose Pseudo-Inverse of $\vec{S}$.Now, calculating eigen value of $\vec{M}\vec{M}^T$,
\begin{align}
\mydet{\vec{M}\vec{M}^T - \lambda\vec{I}} &= 0\\
\implies\myvec{1-\lambda&0&\frac{1}{2}\\0&1-\lambda&1\\\frac{1}{2}&1&\frac{205}{324}-\lambda} &=0\\
\implies-\lambda^3+\frac{853}{324} \lambda^2-\frac{529}{324}\lambda &=0
\end{align}
Hence eigen values of $\vec{M}\vec{M}^T$ are,
\begin{align}
\lambda_1 &=\frac{529}{324}\\
\lambda_2 &= 1\\
\lambda_3 &=0
\end{align}
Hence the eigen vectors of $\vec{M}\vec{M}^T$ are,
\begin{align}
\vec{u}_1=\myvec{\frac{-252}{205}\\\frac{54}{205}\\1},
\vec{u}_2=\myvec{\frac{3}{4}\\1\\0},
\vec{u}_3=\myvec{\frac{7}{9}\\\frac{-1}{6}\\1}
\end{align}
Normalizing the eigen vectors we get,
\begin{align}
\vec{u}_1=\myvec{\frac{-252}{23\sqrt{205}}\\\frac{54}{23\sqrt{205}}\\\frac{\sqrt{205}}{{23}}},
\vec{u}_2=\myvec{\frac{3}{\sqrt{205}}\\\frac{14}{\sqrt{205}}\\0},
\vec{u}_3=\myvec{\frac{14}{23}\\\frac{-3}{23}\\\frac{18}{23}}
\end{align}
Hence we obtain $\vec{U}$ of \eqref{eq100} as follows,
 $\vec{U}$\begin{align}\label{eqU}
\myvec{\frac{-252}{23\sqrt{205}}& \frac{3}{\sqrt{205}}&\frac{14}{23}\\
\frac{54}{23\sqrt{205}}&\frac{14}{\sqrt{205}}&-\frac{3}{23}\\
\frac{\sqrt{205}}{{23}}&0&\frac{18}{23}}
\end{align}
After computing the singular values from eigen values $\lambda_1, \lambda_2, \lambda_3$ we get $\vec{S}$ of \eqref{eq100} as follows,
\begin{align}\label{eqS}
\vec{S}=\myvec{\frac{23}{18}&0\\0&1\\0&0}
\end{align}
Now, calculating eigen value of $\vec{M}^T\vec{M}$,
\begin{align}
\mydet{\vec{M}^T\vec{M} - \lambda\vec{I}} &= 0\\
\implies\myvec{\frac{130}{81}-\lambda&\frac{-7}{54}\\\frac{-7}{54}&\frac{37}{36}-\lambda} &=0\\
\implies\lambda^2-\frac{853}{324}\lambda+\frac{529}{324} &=0
\end{align}
Hence eigen values of $\vec{M}^T\vec{M}$ are,
\begin{align}
\lambda_4 &= \frac{529}{324}\\
\lambda_5 &=1
\end{align}
Hence the eigen vectors of $\vec{M}^T\vec{M}$ are,
\begin{align}
\vec{v}_1=\myvec{\frac{-14}{3}\\1},
\vec{v}_2=\myvec{\frac{3}{14}\\1}
\intertext{Normalizing the eigen vectors we get,}
\vec{v}_1=\myvec{\frac{-14}{\sqrt{205}}\\\frac{3}{\sqrt{205}}},
\vec{v}_2=\myvec{\frac{3}{\sqrt{205}}\\\frac{14}{\sqrt{205}}}
\end{align}
Hence we obtain $\vec{V}$ of \eqref{eq100} as follows,
\begin{align}
\vec{V}=\myvec{\frac{-14}{\sqrt{205}}&\frac{3}{\sqrt{205}}\\\frac{3}{\sqrt{205}}&\frac{14}{\sqrt{205}}}
\end{align}
 From \eqref{eq100} we get the Singular Value Decomposition of $\vec{M}$ ,
\begin{align}
\vec{M} = \myvec{\frac{-252}{23\sqrt{205}}& \frac{3}{\sqrt{205}}&\frac{14}{23}\\
\frac{54}{23\sqrt{205}}&\frac{14}{\sqrt{205}}&-\frac{3}{23}\\
\frac{\sqrt{205}}{{23}}&0&\frac{18}{23}}\myvec{\frac{23}{18}&0\\0&1\\0&0}\myvec{\frac{-14}{\sqrt{205}}&\frac{3}{\sqrt{205}}\\\frac{3}{\sqrt{205}}&\frac{14}{\sqrt{205}}}^T
\end{align}
Moore-Penrose Pseudo inverse of $\vec{S}$ is given by,
\begin{align}
\vec{S_+} = \myvec{\frac{18}{23}&0&0\\0&1&0}
\end{align}
From \eqref{eqX} we get,
\begin{align}
\vec{U}^T\vec{b}&=\myvec{-\frac{415107}{100000}\\\frac{167623}{50000}\\-\frac{154347}{50000}}\\
\vec{S_+}\vec{U}^T\vec{b}&=\myvec{-\frac{1624313691}{500000000}\\\frac{167623}{50000}}\\
\vec{x} = \vec{V}\vec{S_+}\vec{U}^T\vec{b} &= \myvec{\frac{969580693}{250000000}\\\frac{32467791}{12500000}}=\myvec{3.87\\2.59}\label{eq85}
\end{align}
Verifying the solution of \eqref{eq85} using,
\begin{align}
\vec{M}^T\vec{M}\vec{x} = \vec{M}^T\vec{b}\label{eqVerify}
\end{align}
Evaluating the R.H.S in \eqref{eqVerify} we get,
\begin{align}
\vec{M}^T\vec{M}\vec{x} &= \myvec{\frac{53}{9}\\\frac{13}{6}}\\
\implies\myvec{\frac{130}{81}&\frac{-7}{54}\\\frac{-7}{54}&\frac{37}{36}}\vec{x} &= \myvec{\frac{53}{9}\\\frac{13}{6}}\label{eq:eq17}
\end{align}
Solving the augmented matrix of \eqref{eq:eq17} we get,
\begin{align}
\myvec{\frac{130}{81}&\frac{-7}{54}&\frac{53}{9}\\\frac{-7}{54}&\frac{37}{36}&\frac{13}{6}} &\xleftrightarrow{R_1=\frac{81}{130}R_1}\myvec{1&\frac{-21}{260}&\frac{477}{130}\\\frac{-7}{54}&\frac{37}{36}&\frac{13}{6}}\\
&\xleftrightarrow{R_2=R_2+\frac{7}{54}R_1}\myvec{1&\frac{-21}{260}&\frac{477}{130}\\0&\frac{529}{520}&\frac{687}{260}}\\
&\xleftrightarrow{R_2=\frac{520}{529}R_2}\myvec{1&\frac{-21}{260}&\frac{477}{130}\\0&1&\frac{1374}{529}}\\
&\xleftrightarrow{R_1=R_1+\frac{21}{260}R_2}\myvec{1&0&\frac{2052}{529}\\0&1&\frac{1374}{529}}\label{eq:eq13}
\end{align}
From equation \eqref{eq:eq13}, solution is given by,
\begin{align}\label{eq:eq14}
\vec{x}=\myvec{\frac{2052}{529}\\\frac{1374}{529}}=\myvec{3.87\\2.59}
\end{align}
Comparing results of $\vec{x}$ from \eqref{eq85} and \eqref{eq:eq14}, we can say that the solution is verified.
\end{document}
 